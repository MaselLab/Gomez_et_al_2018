\documentclass[11pt,one column]{article}
\usepackage[margin=.5in]{geometry}
\usepackage{amsmath}
\usepackage{amssymb}
\usepackage{graphicx}
\usepackage{capt-of}

\graphicspath{{figures/}}

\begin{document}
\section*{Introduction}
\begin{itemize}
\item Natural selection frequently acts on multiple traits simultaneously. 
\begin{itemize}
\item A single trait’s response to selection leads to a shift in the population mean towards the fittest phenotype.
\item The response to selection with multiple traits involves both direct and indirect effects that together determine their evolution \cite{Scarcelli23102007,Lovell2013,Wagner2011}.  
\item Indirect effects originate from genetic mechanisms that underlie trait expression, which may be associated or shared in a manner that enforces specific patterns of co-expression.
\item Darwin, while this fact, recognized that such associations likely played an important role in directing the evolution of organisms under natural selection (Darwin p. 130-133).
\item Presently, we know that these genetic associations can fundamentally alter how populations respond to selection and direct their adaptation towards certain evolutionary outcomes, such as speciation or extinction \cite{Felsenstein1979, Arnold2001, Arnold2008}.
\item This makes them important to many micro- and macro-evolutionary questions central to evolutionary biology.
\end{itemize}

\item On short time scales, selection on phenotypes composed of multiple quantitative traits is best understood using the framework of quantitative genetics. 
\begin{itemize}	
\item The strength of a genetic associations between traits are estimated and summarized by genetic correlations, or genetic covariances.
\item Genetic covariances form part of the components of the \textbf{G} matrix.
\item The response of phenotypes to selection is mathematically described by Lande's equation ($\Delta \bar{z} = \textbf{G} \hspace{.05in}\beta$).
\item Here, $\Delta \bar{z}$ is the response of the mean of each trait, and $\beta$ is the selection gradient on the set of traits. 
\item Lande’s formulation provides a means of estimating responses to selection correlated traits, but it does not indicate gives rise to them.
\item Trait correlations can generally come about for a number of reasons, but they often have a genetic basis \cite{Saltz2017} arising from either pleiotropy, linkage disequilibria, or trans acting elements.
\item Most evolutionary biologist agree that the first two are much more important.
\end{itemize}

\item Historically, there have been two schools of thought on which of the first two is responsible for most of the genetic correlations.
\begin{itemize} 
\item The first of these was put forth by the Edinburgh school and held that pleiotropy was primary cause of most genetic correlations.
\item This belief was largely motivated by their interests in relationships between parent-offspring phenotypes within randomly mating populations and their implication for their evolution.
\item In these sorts of populations, linkage disequilibrium can rapidly decay by recombination on evolutionary time scales (\cite{fox2006evolutionary}, chp 20), while pleiotropic effects tend to persist.
\item The alternative view which came out of the Birmingham school held that linkage disequilibria was much more significant.  
\item Their interest led them to experiments involving inbred lines with plants where the genetic correlations from linkage disequilibria were just as substantial.
\item Inarguably, the question of which view is perhaps correct largely depends the details of the population.
\item 
\end{itemize}

\item As populations evolve, the components of the \textbf{G}-matrix change as well, which may constrain or fascilitate adaptation.
\begin{itemize}
\item The question of how stable the \textbf{G}-matrix is over time, is directly related to the question of how genetic correlations evolve, and that of course, depends on their causes.
\item Today most geneticists assume that pleiotropy underlies many of the genetic correlations we observe directing adaptation, and as a result, have developed many sophisticated models to examine how genetic correlations arising from .
\item We focus asexauls micro-organisms, which include some Eukaryotes and all Prokaryotes, experience very little recombination of their DNA or lack it entirely.	
\item The fact the majority of life on the planet falls into this category underscores the Birmingham view to some degree.
\end{itemize}

\item The evolution of the 
\begin{itemize}
\item We focus on asexual micro-organisms that lack recombination and examine how the evolution of their traits is affected by linkage.	
\item In the case of adaptive traits, with fitness conferred taken as their quantitative measure, the relationship between variance and the population parameters is known explicitly. 
\item The work of Desai and Fisher (2007) provide the variance in fitness explicitly shown to work of gives the variance of adaptive traits in terms of the population’s parameters $N$, $U_b$, and $s$. 
\item The variance of traits and the covariance between them are estimated, and assume that the underlying traits satisfy basic assumptions, such as being polygenic. 
\item The variance in fitness was not something that could be determine from populations parameters, until only recently. 
\item Desai and Fisher’s work (2007) provides the rate of adaptation of the population in terms of its parameters, \[v =s^2  \frac{2 \ln(Ns)-\ln(s/U_b)}{\ln^2(s/U_b)}\] and since the rate of adaptation is the variance of fitness ($v=\sigma^2$).
\item If we consider a second adaptive trait evolving concurrently, with identical mutation rate and selection coefficient, then the rate of adaptation in each trait should be half of the total rate of adaptation $v(2U_b)$ since evolution in each contributes equally the overall adaptation of the population. 
\item A simple application of Desai and Fisher’s formula quickly shows that the rate of adaptation for a single trait is reduced when a second trait is considered since \[ 1/2  v(2U)<v(U). \]
\item Naturally, we should expect such a result since we have doubled the mutation rate, and consequently, increased the amount of clonal interference that occurs within the population. 
\item However, the mere fact that we are also considering a second trait indicates that there to this picture than this.
\item The rate of adaptation in one trait is determined by its variance and covariance with other traits, and this relation is expressed in the breeder’s equation. In our situation, both quantitative traits are adaptive and we measure them by the fitness they endow the organism. Consequently, the rate of fitness increase of the focal trait $v_1$ must be related \[v_1=\sigma_1^2+\sigma_{12}.\]
\end{itemize}

\item (\textbf{Approach}) To examine the effects on the evolution of one trait by adaptation in another, we develop a model of a traveling wave over a two dimensional trait space.
\begin{itemize}
\item 
\end{itemize}

\end{itemize}

\section*{Methods}
\begin{itemize}
\item In our model we consider a population of asexuals at carrying capacity $N$ evolving in a static environment.
\begin{itemize}
\item Individuals have two traits contributing to their relative fitness which are controlled by separately loci that independently evolve.
\item Beneficial mutations contribute fitter alleles to the genetic diversity of each and occur at rate $U_k$ (trait $k=1,2$); deleterious mutations are not considered.
\item Each new mutation in trait \textit{k} additively increases the trait's contribution to an individual's fitness by an amount $s_k$.
\item An individual's overall fitness is $r_{ij} = i s_1+j s_2$, if it has accumulated $i$ mutations in the first trait and $j$ in the second.
\item Its relative fitness is $(r_{ij}-\bar{r})$, where $\bar{r} = \bar{i} s_1+\bar{j} s_2$ is the population mean fitness, and $\bar{i}$ and $\bar{j}$ are the number of mutations in the average individual.
\item We divide the population into classes composed of individuals with an identical number of beneficial mutations in each trait, and denote their abundances with $n_{ij}$ and frequencies with $p_{ij}$.
\end{itemize}	

\item We focus on the evolutionary regime characterized by large populations with strong selection for which $1/N \ll s \ll 1$.
\begin{itemize}
\item This allows us to categorize the set of abundances $n_{ij}$ into two groups.
\item Classes in the first group are dominated by selection, and we refer to this group as the bulk following the convention of Desai and Fisher (2007).
\item Abundances of the remaining set of classes are largely affected by drift and flucuate randomly as a result.
\item We refer to this group as the stochastic front.
\end{itemize}

\item The bulk behaves nearly deterministic over time and we can approximate its behavior with a system of ordinary differential equations.
\begin{itemize}
\item Abundances in the bulk change in proportion to their selective advantage and accordingly their dynamics follow the ODE \[ \dot{n}_{ij}(t) = r_{ij} n_{ij}(t). \]
\end{itemize}

\item (\textbf{Stochastic Front})
\begin{itemize}
\item As for the one-dimensional traveling wave, the key variable of interest is the time-between establishments $\tau$ of new beneficial mutations. 
\item This random variable summarizes all of the stochastic effects that accumulate over time during the growth of a fitness class, and determines the rate at which fitness classes in the stochastic front transition into the deterministic regime where selection is the prominent force. 
\item We will be interested in the rate of these transitions, because they will determine two very important aspects of the two-dimensional distribution’s evolution that influence trait evolution. 
\item The probability of the latter event, denoted $\pi_{fix}$, is proportional to the selective advantage of the mutation. In these cases a beneficial mutation is said to have established in the population. 
\item The threshold separating these two characteristically different behaviors in growth/decline occurs at size equal to the inverse of the class' probability of fixation.
\end{itemize}

\item (\textbf{Rate of Adaptation})
\begin{itemize}
\item In the case of adaptive traits, with fitness conferred taken as their quantitative measure, the relationship between variance and the population parameters is known explicitly. 
\item The work of Desai and Fisher (2007) provide the variance in fitness explicitly shown to work of gives the variance of adaptive traits in terms of the population’s parameters $N$, $U_b$, and $s$. 
\item From Eq. 3, the rate of adaptation can be expressed as the sum of the individual rates of adaptation in each trait, indicated by subscripts, as $\dot{\bar{r}}=\dot{\bar{r}}_1 +\dot{\bar{r}}_2$.
\item The variance of traits and the covariance between them are estimated, and assume that the underlying traits satisfy basic assumptions, such as being polygenic. 
\item The variance in fitness was not something that could be determine from populations parameters, until only recently. 
\item Desai and Fisher’s work (2007) provides the rate of adaptation of the population in terms of its parameters, \[v =s^2  \frac{2 \ln(Ns)-\ln(s/U_b)}{\ln^2(s/U_b)}\] and since the rate of adaptation is the variance of fitness ($v=\sigma^2$).
\item If we consider a second adaptive trait evolving concurrently, with identical mutation rate and selection coefficient, then the rate of adaptation in each trait should be half of the total rate of adaptation $v(2U_b)$ since evolution in each contributes equally the overall adaptation of the population. 
\item A simple application of Desai and Fisher’s formula quickly shows that the rate of adaptation for a single trait is reduced when a second trait is considered since \[ 1/2  v(2U)<v(U). \]
\item Naturally, we should expect such a result since we have doubled the mutation rate, and consequently, increased the amount of clonal interference that occurs within the population. 
\end{itemize}

\item (\textbf{Covariance and Trait Evolution})
\begin{itemize}
\item The results for one-dimensional traveling waves provides the relationship between variance in fitness from one trait evolving and population parameters $N$, $U_k$, and $s_k$. 
\item Mutation-selection balance induces steady state values for the speed of evolution in the two, from which it possible to determine the two variances, the single covariance, and furthermore, the resulting G-matrix of the two traits (Eq. 5). \[ G= \left( \begin{array}{cc} \sigma_1^2& \sigma_{1,2}\\ \sigma_{12} & \sigma_2^2 \end{array} \right) \]
\item However, the mere fact that we are also considering a second trait indicates that there to this picture than this.
\item The rate of adaptation in one trait is determined by its variance and covariance with other traits, and this relation is expressed in the breeder’s equation. In our situation, both quantitative traits are adaptive and we measure them by the fitness they endow the organism. Consequently, the rate of fitness increase of the focal trait $v_1$ must be related \[v_1=\sigma_1^2+\sigma_{12}.\]
\item Each of the two individual rates are related to the variance in fitness from the respective trait and its covariance in fitness with other (Eq. 4). \[ \dot{\bar{r}}_1=\sigma_1^2+\sigma_{12}\] \[ \dot{\bar{r}}_2=\sigma_2^2+\sigma_{12} \]
\item We examine how genetic correlations arise and influence the evolution of traits in adapting populations, we focus on regimes of adaptation characterized by significant levels of genetic variation in each trait. 
\item Linkage disequilibria continually arises throughout the adaptation process and limits the response of traits to selection. In the case of two alleles, linkage disequilibria is a measure of the statistical association between them. 
\item In turn, the covariance in fitness can be written out as Eq. 6. \[ \sigma_{12}=\sum_{ij}D_{ij}(i-\bar{i})s_1 (j-\bar{j}) s_2 \] and it becomes evident that linkage disequilibrium directly affects the evolution of the two traits.
\item We can also note from this expression that making a distinction between alleles that yields the same fitness advantage does not change the covariance.
\end{itemize}

\item (\textbf{Simulation description}) We examined the resulting dynamics of trait evolution using numerical simulations of our model carried out in Mathematica.
\begin{itemize}
\item We focused on modeling the symmetric case, in which the two traits had identical mutation rate and selection coefficients.
\item These, along with the population size, were chosen to match values reported in evolutionary experiments involving relevant micro-organisms (see Table 1).  
\item We also focused parameter sets in which $1/N \ll s \ll 1$ to ensure that changes in abundances for the population bulk could be treated separately from those in the stochastic front.
\end{itemize}

\item Our numerical implementation consisted of two routines which ran iteratively for the duration of the simulation.
\begin{itemize}
\item Each iteration began with solving for the time-dependent abundances in the bulk governed by equation [??] over a time interval of one-thousand generations.
\item We used Mathematica's standard numerical ODE solver to obtain the solutions and used them to determine which class established next and the time at which it did so.
\item Abundances in the bulk were then examined at the time of establishment.
\item Any classes in the bulk with less than one individual were removed, while abundances were updated to their size at the time of establishment for those that weren't.
\item The establishing class was then added to the bulk with initial size randomly determined using equation [??].
\item The next iteration then began with solving for the abundances of the new bulk, followed by a repetition of the steps just mentioned.  
\end{itemize}	

\item We simulated the time of the next establishment by taking the first mutant whose lineage manages to fix. 
\begin{itemize}
\item The times that mutants appear follow a time-inhomogeneous Poisson process with rate functions given by the abundances of the adjacent classes from which they originated. 
\item Consequently, if $\tau_l$ denotes the arrival time of the $l$-th mutant generated from the class $n_{ij}(t)$ with an additional mutation in trait $k$, then \[ \int_0^{\tau_l} U_k n_{ij}(t) dt \] has the distribution of $l$ i.i.d exponentially distributed summed random variables. 
\item We began with identifying the first mutant to appear by solving for the set of $\tau_1$ 's of each class in the stochastic front.  
\item The probability of fixation for this mutant was calculated and used to check if the the mutant's lineage would establish.  
\item If mutation did establish, then the class was incorporated into the bulk as described above.  
\end{itemize}

\item Simulations were initialized with monoclonal populations at carrying capacity consisting of individuals having no beneficial mutations in either trait and relative fitness set zero.
\begin{itemize}
\item We allowed the simulation to run for 5,000 generations to allow the population to achieve beneficial mutation-selection balance before collecting data.
\item This ensured that the transient effects from the initializing would not distort the statistics collected for the distribution.
\item Following the burn period of the simulation, we recorded the existing classes in the bulk and their respective abundances every one 100 generations and intermittent times at which new classes established.
\item The data was used to compute the marginal mean fitness and variances of each trait, as well as their covariance.
\item From these quantities we also computed the instantaneous rate of adaptation (eqtn. ??), the mean rate of fitness increase in each trait and the overall rate of adaptation.
\end{itemize}

\end{itemize}

% figure to include in methods
\begin{figure}[h]
	\centering
	\includegraphics[width=0.35\linewidth]{2dfitnessdistributionexample.png} 
	\caption{Two-dimensional fitness distribution.} 
	\label{fig.1}
\end{figure}
	
\begin{table}[h]
	\centering
 	\begin{tabular}{ | c | c | c | c |}
    \hline
    Organism 												& Population Size 	& Selection Coefficient & Beneficial Mutation Rate \\ \hline
    \textit{E}. coli \cite{Perfeito2007}					& $10^9 - 10^{12}$ 	& $0.5 - 2 \%$  & $10^{-5} - 10^{-4}$ \\ \hline
    \textit{S}. cerevisiae \cite{desai2007speed,Levy2015}	& $10^9 - 10^{12}$ 	& $0.5 - 2 \%$  & $10^{-5} - 10^{-4}$ \\
    \hline
	\end{tabular}
	\caption{Population parameters measured in evolution experiments for key microorganisms.} 
	\label{Table 1}
\end{table}


\section*{Results}
\begin{itemize}
\item Mean trait variances and covariances over the period of the simulation
\item Fluctuations in the components of the \textbf{G}-matrix.
\item 
\end{itemize}

%figures to include in results
\begin{figure}[h]
\centering
\includegraphics[width=0.35\linewidth]{fig2_N-10p09_c1-0d01_c2-0d01_U1-1x10pn5_U2-1x10pn5_exp1}
\caption{Average rate of adaptation, variance and covariance. Parameters $N=10^9$, $s_1=s_2=0.01$, $U_1=U_2=10^{-5}$}
\label{fig.1}
\end{figure}

\begin{figure}[h]
\centering
\includegraphics[width=0.35\linewidth]{fig1_N-10p09_c1-0d01_c2-0d01_U1-1x10pn5_U2-1x10pn5_exp1}
\captionof{figure}{Variance and covariance fluctuations over time. Parameters $N=10^9$, $s_1=s_2=0.01$, $U_1=U_2=10^{-5}$}
\label{fig.2}
\end{figure}

\section*{Discussion}
\begin{itemize}
\item Negative covariance driven by LD
\item \textbf{G}-matrix instability and its implication for the long term evolution of populations
\end{itemize}

\bibliographystyle{plain}
\bibliography{biblio}

\section*{Appedices}
\begin{itemize}
\item relationship between LD and covariance
\end{itemize}


\end{document}