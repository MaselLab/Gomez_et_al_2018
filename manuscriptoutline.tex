\documentclass[12pt,one column]{article}
\usepackage[margin=.5in]{geometry}
\usepackage{amsmath}
\usepackage{amssymb}

\begin{document}
\section*{Introduction}
\begin{itemize}
\item What are we talking about? 
	\begin{itemize}
	\item Trait correlations (TC) and their role in adaptive evolution \cite{Scarcelli23102007,Lovell2013,Wagner2011}
	\end{itemize}
\item What do we already know about trait correlations?
	\begin{itemize}
	\item TC in the minds of evolutionary biologist.
	\item Influential works on TC and their insights. 
	\item Advances in genetics and the Landau's breeder equation.
	\item The causes of TC and how they shape them over time over time.
	\item The Edinburgh view of trait correlations and adaptation: F.A.E. Crew, C.H. Waddington, D.S. Falconer, F.W. Robertson  
	\begin{itemize}
		\item Edinburgh school on animal genetics and evolution allowed for higher heterozygosity and number of alleles
	\end{itemize}
	\item The Birmingham view of trait correlations and adaptation: Mathers, Jinks, Kearsey, Pooni
	\begin{itemize}
		\item Birmingham school dealt with almost pure inbred plant lines, with only two alleles at each segregating locus
	\end{itemize}
	\end{itemize}
\item Why are we talking about this subject?
	\begin{itemize}
	\item Pleiotropy based models for G-matrix evolution already exist.
	\item LD's role in shaping TC has not been explored despite being necessary.
	\end{itemize}
\item What are we going to say about the 
	\begin{itemize}
	\item Trait correlations and their role in adaptive evolution
	\end{itemize}
\end{itemize}

\section*{Results}
\begin{itemize}
	\item one
\end{itemize}

\section*{Discussion}
\begin{itemize}
	\item
\end{itemize}

\section*{Methods}
\begin{itemize}
	\item
\end{itemize}

\bibliographystyle{plain}
\bibliography{biblio}

\end{document}