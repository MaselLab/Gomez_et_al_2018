\documentclass[11pt,one column]{article}
\usepackage[margin=.5in]{geometry}
\usepackage{amsmath}
\usepackage{amssymb}
\usepackage{graphicx}
\usepackage{capt-of}

\graphicspath{{figures/}}

\begin{document}
\section*{Introduction}
\begin{itemize}
\item Natural selection frequently acts on multiple traits simultaneously. 
\begin{itemize}
\item A single trait's response to selection leads to a shift in the population mean towards the fittest phenotype.
\item With multiple traits, the response involves both direct and indirect effects that together determine the evolution each \cite{Scarcelli23102007,Lovell2013,Wagner2011}.  
\item Indirect effects originate from genetic mechanisms that underlie trait expression, which may be associated or shared in a manner that enforces specific patterns of co-expression.
\item These associations can alter fundamental features of adaptation and have significant evolutionary consequences \cite{Felsenstein1979, Arnold2001, Arnold2008}.
\end{itemize}

\item For quantitative traits, genetic associations and their effects on adaptation are mathematically described by Landau's equation (($\Delta \bar{z} = \textbf{G} \hspace{.05in}\beta$))
\begin{itemize}	
\item It states that the change in the population mean $\Delta \bar{z}$ equals the product of the selection gradient $\beta$ and what is known as the \textbf{G}-matrix, whose components are the additive genetic covariances between traits.
\item Lande’s result has been used to extend much of our understanding of how phenotypes evolve under selection. 
\item When combined with the notions of adaptive landscapes, a rich set of ideas emerge that emphasize the role of the \textbf{G}-matrix in directing adaptive evolution at both micro- and macro-evolutionary scales.
\item This is one reason why the evolution of the \textbf{G}-matrix has been of key interest to evolutionary biologist. 
\item Furthermore, these ideas make it apparent that the evolution of traits cannot understood in isolation, and in fact must be considered alongside what we know about the dynamics of the \textbf{G}-matrix.
\end{itemize}

\item The behavior of the \textbf{G}-matrix over time largely depends on what are the causes of the genetic correlations.
\begin{itemize}
\item Correlations between quantitative traits can arise for a number of reasons, but they often have a genetic basis \cite{Saltz2017}.
\item They arise from either pleiotropy, linkage disequilibria, or are the product of trans acting elements; the first two that are thought to be the most significant factors.
\item Historically, there have been two schools of thought what the principle causes of genetic correlations are.
\item The first was put forth by the Edinburgh school, which held that pleiotropy was principle source, since linkage disequilibrium could rapidly decay by recombination (\cite{fox2006evolutionary}, chp 20).
\item This belief was largely motivated by interests the relationships between parent-offspring phenotypes within randomly mating populations, and its implications for evolution.
\item The alternative view came out of the Birmingham school, which largely experimented with inbred lines of plants.
\item Geneticists in this group believed that linkage disequilibria was much more important; a fact that often reflected in their experiments.
\item Presently, the Edinburgh view is more widely accepted, and as a result, most models for the evolution of the \textbf{G}-matrix assume genetic correlations are pleiotropic proceed with examining how evolutionary processes influence them.
\item We know far less about the about the behavior genetic correlations from linkage disequilibria over time, but this is not much of an issue for most complex organisms.
\end{itemize}

\item In the case of asexuals, the Birmingham view is more appropriate, and determining how the genetic correlations due to LD change over time is critical to our understanding of adaptation.
\begin{itemize}
\item Linkage disequilibria persists for much longer periods of time in a asexuals, which can adversely affect adaptation.
\item This fact has been used to argue for the evolutionary advantages of sex and recombination \cite{Barton2005}.
\item Consequently, we can learn a substantial amount about the evolution of their traits by examining how genetic correlations from LD evolve over time and influence adaptation.
\item Do so would not only reveal a great deal about the oldest and most abundant forms of life on the planet, such as bacteria, but could prove useful to our use of them in evolutionary experiments.
\end{itemize}

\item Our best models for adaptation in asexuals make use of what are known as traveling waves, and are based on population genetics.
\begin{itemize}
\item They have been used to establish important relationships between the population's parameter and key features of evolution in asexuals.
\item Among these a model of Desai and Fisher \cite{desai2007beneficial}, which they used to derive an expression for the rate of adaptation, given by $v(N,s,U) \approx 2s^2 \ln(N s)/\ln^2(s/U)$.
\item Here, $N$ is the population size, $s$ is the effective selective advantage of a beneficial mutation, and $U$ is the beneficial mutation rate.
\item This is an important result, because it directly provides the additive variance in fitness by applying Fisher's fundamental theorem ($v=\sigma^2$).
\item If we regard the population as having one adaptive trait, then this also equals the trait's additive variance and realize that we have said something more about the quantitative genetics involved.
\end{itemize}

\item Unfortunately, if we apply Desai and Fisher's result to the case of two adaptive traits, we still can still solve for the overall fitness variance $\sigma^2$ of the population, but nothing can be said about the variances and covariance of the two traits.
\begin{itemize}
\item This highlights the shortcomings of the traveling wave model in the case of multiple traits, for which we should expect Landau's equation to hold.
\item As an example, consider a single adaptive trait evolving with parameters $N$, $s$ and $U$, then its variance and rate of adaption are $v(U;N,s)$ according to the formula given above.
\item If the population acquires a second adaptive trait with identical parameters $s$ and $U$, then the new rate of adaptation is clearly $v(2U;N,s)$.
\item By symmetry we know that rate of adaptation in each trait alone must be $0.5 \hspace{.02 in} v(2U;N,s)$, and furthermore, the rate of adaptation in trait one is now less with the second trait.
\item Clearly, we lack any information about how trait two is affecting trait one's evolution, but Landau's equation tells us that these effects must somehow translate into changes in the additive variance of trait one and its covariance with the second.
\end{itemize}

\item To examine how adaptation in one trait is affected by evolution in another, we extend the work of Desai and Fisher's traveling wave model and consider a two dimensional trait space.
\begin{itemize}
\item We focus on the simplest case in which individuals have two independently evolving traits contributing to fitness.
\item Our aim is to explore the effects of genetic correlations arising from linkage disequilibrium contribute to the interactions that direct trait evolution.
\item So, we assume that the two traits are controlled separate non-pleiotropic loci.
\item Our exploration of the question will done using simulations, whose results we present after discussing key details of the model in the next section.
\item We will conclude with a discussion of the our findings and their implications.
\end{itemize}

\end{itemize}

\section*{Methods}
\begin{itemize}
\item We consider an asexually reproducing population at carrying capacity $N$ evolving in a static environment.
\begin{itemize}
\item Individuals have two traits that contribute to their relative fitness, each controlled by separate loci that independently evolve.
\item Beneficial mutations occur at a rate of $U_k$ (trait $k=1,2$); deleterious mutations are not considered.
\item Each mutation additively increases the trait's contribution to an individual's fitness by an amount $s_k$.
\item If the individual has accumulated $i$ mutations in the first trait and $j$ in the second, then its overall fitness is $r_{ij} = i s_1+j s_2$.
\item Its relative fitness is then $(r_{ij}-\bar{r})$, where $\bar{r} = \bar{i} s_1+\bar{j} s_2$ is the population mean fitness, with $\bar{i}$ and $\bar{j}$ being the number of mutations found in the average individual.
\item The population is divided into groups composed of individuals with an identical number of beneficial mutations in each trait; we will refer to them as classes.
\item We denote their abundances by $n_{ij}$ and their frequencies by $p_{ij}$.
\end{itemize}	

\item We assumed that our population was large and evolving in a strong-selection strong-mutation regime, which required that the population parameters satisfied $1/N \ll s_k \ll 1$ and $N\hspace{.02in}U_k \gg 1$.
\begin{itemize}
\item The first relation provides a separation of deterministic and stochastic behavior in abundances.
\item This allowed use to consider the dynamics of the population as two interacting pieces, which greatly simplified formulation and implementation of the model.
\item The second was necessary to ensure variation in each trait, which is a critical aspect of the question we aimed to answer.
\item Following the work of Desai and Fisher (2007), we refer to set of classes whose abundances behave deterministically as the bulk.
\item The set of remaining classes whose abundances behave stochastically due to drift are referred to as the stochastic front.
\end{itemize}

\item The dynamics of the bulk are described by a system of ordinary differential equations.
\begin{itemize}
\item Abundances change in proportion to their selective advantage, and consequently, their behavior can be approximated by ODE 
\begin{equation} 
\dot{n}_{ij}(t) = (r_{ij}-\bar{r}) n_{ij}(t). 
\end{equation}
\item The population's size remains roughly constant over time as a result.
\item Moreover, our assumptions concerning $N$ and $s$ imply that the sum of abundances in the stochastic front are negligible in comparison to the population's size.
\item This allows us to model changes in the $\{p_{ij} \}$ for the bulk with an analogous ODE to equation 1.
\item Additive variances and covariance, which tell us about how traits influence the evolution of one another, are entirely determined by these frequencies.
\end{itemize}

\item Classes in the stochastic front intermittently transition from stochastic to deterministic behavior and become part of the bulk.
\begin{itemize}
\item This occurs when a class has grown sufficiently large so as to escape the effects of drift.
\item The key aspects of this process are the rate at which it occurs and initial size of classes once they transition; both are entirely determined by the parameters $N$, $U_k$ and $s_k$. 
\item The stochastic growth of classes in the front follow a branching process fed by incoming mutants generated by adjacent less fit classes in the bulk.
\item Each of these continue to exhibit stochastic fluctuations until it has reached a size that 
\end{itemize}

\item (\textbf{Rate of Adaptation})
\begin{itemize}
\item In the case of adaptive traits, with fitness conferred taken as their quantitative measure, the relationship between variance and the population parameters is known explicitly. 
\item The work of Desai and Fisher (2007) provide the variance in fitness explicitly shown to work of gives the variance of adaptive traits in terms of the population’s parameters $N$, $U_b$, and $s$. 
\item From Eq. 3, the rate of adaptation can be expressed as the sum of the individual rates of adaptation in each trait, indicated by subscripts, as $\dot{\bar{r}}=\dot{\bar{r}}_1 +\dot{\bar{r}}_2$.
\item The variance of traits and the covariance between them are estimated, and assume that the underlying traits satisfy basic assumptions, such as being polygenic. 
\item The variance in fitness was not something that could be determine from populations parameters, until only recently. 
\item Desai and Fisher’s work (2007) provides the rate of adaptation of the population in terms of its parameters, \[v =s^2  \frac{2 \ln(Ns)-\ln(s/U_b)}{\ln^2(s/U_b)}\] and since the rate of adaptation is the variance of fitness ($v=\sigma^2$).
\item If we consider a second adaptive trait evolving concurrently, with identical mutation rate and selection coefficient, then the rate of adaptation in each trait should be half of the total rate of adaptation $v(2U_b)$ since evolution in each contributes equally the overall adaptation of the population. 
\item A simple application of Desai and Fisher’s formula quickly shows that the rate of adaptation for a single trait is reduced when a second trait is considered since \[ 1/2  v(2U)<v(U). \]
\item Naturally, we should expect such a result since we have doubled the mutation rate, and consequently, increased the amount of clonal interference that occurs within the population. 
\end{itemize}

\item (\textbf{Covariance and Trait Evolution})
\begin{itemize}

\item The results for one-dimensional traveling waves provides the relationship between variance in fitness from one trait evolving and population parameters $N$, $U_k$, and $s_k$. 
\item Mutation-selection balance induces steady state values for the speed of evolution in the two, from which it possible to determine the two variances, the single covariance, and furthermore, the resulting G-matrix of the two traits (Eq. 5). \[ G= \left( \begin{array}{cc} \sigma_1^2& \sigma_{1,2}\\ \sigma_{12} & \sigma_2^2 \end{array} \right) \]
\item However, the mere fact that we are also considering a second trait indicates that there to this picture than this.
\item The rate of adaptation in one trait is determined by its variance and covariance with other traits, and this relation is expressed in the breeder’s equation. In our situation, both quantitative traits are adaptive and we measure them by the fitness they endow the organism. Consequently, the rate of fitness increase of the focal trait $v_1$ must be related \[v_1=\sigma_1^2+\sigma_{12}.\]
\item Each of the two individual rates are related to the variance in fitness from the respective trait and its covariance in fitness with other (Eq. 4). \[ \dot{\bar{r}}_1=\sigma_1^2+\sigma_{12}\] \[ \dot{\bar{r}}_2=\sigma_2^2+\sigma_{12} \]
\item We examine how genetic correlations arise and influence the evolution of traits in adapting populations, we focus on regimes of adaptation characterized by significant levels of genetic variation in each trait. 
\item Linkage disequilibria continually arises throughout the adaptation process and limits the response of traits to selection. In the case of two alleles, linkage disequilibria is a measure of the statistical association between them. 
\item In turn, the covariance in fitness can be written out as Eq. 6. \[ \sigma_{12}=\sum_{ij}D_{ij}(i-\bar{i})s_1 (j-\bar{j}) s_2 \] and it becomes evident that linkage disequilibrium directly affects the evolution of the two traits.
\item We can also note from this expression that making a distinction between alleles that yields the same fitness advantage does not change the covariance.
\end{itemize}

\item (\textbf{Simulation description}) We examined the resulting dynamics of trait evolution using numerical simulations of our model carried out in Mathematica.
\begin{itemize}
\item We focused on modeling the symmetric case, in which the two traits had identical mutation rate and selection coefficients.
\item These, along with the population size, were chosen to match values reported in evolutionary experiments involving relevant micro-organisms (see Table 1).  
\item We also focused parameter sets in which $1/N \ll s \ll 1$ to ensure that changes in abundances for the population bulk could be treated separately from those in the stochastic front.
\end{itemize}

\item Our numerical implementation consisted of two routines which ran iteratively for the duration of the simulation.
\begin{itemize}
\item Each iteration began with solving for the time-dependent abundances in the bulk governed by equation [??] over a time interval of one-thousand generations.
\item We used Mathematica's standard numerical ODE solver to obtain the solutions and used them to determine which class established next and the time at which it did so.
\item Abundances in the bulk were then examined at the time of establishment.
\item Any classes in the bulk with less than one individual were removed, while abundances were updated to their size at the time of establishment for those that weren't.
\item The establishing class was then added to the bulk with initial size randomly determined using equation [??].
\item The next iteration then began with solving for the abundances of the new bulk, followed by a repetition of the steps just mentioned.  
\end{itemize}	

\item We simulated the time of the next establishment by taking the first mutant whose lineage manages to fix. 
\begin{itemize}
\item The times that mutants appear follow a time-inhomogeneous Poisson process with rate functions given by the abundances of the adjacent classes from which they originated. 
\item Consequently, if $\tau_l$ denotes the arrival time of the $l$-th mutant generated from the class $n_{ij}(t)$ with an additional mutation in trait $k$, then \[ \int_0^{\tau_l} U_k n_{ij}(t) dt \] has the distribution of $l$ i.i.d exponentially distributed summed random variables. 
\item We began with identifying the first mutant to appear by solving for the set of $\tau_1$ 's of each class in the stochastic front.  
\item The probability of fixation for this mutant was calculated and used to check if the the mutant's lineage would establish.  
\item If mutation did establish, then the class was incorporated into the bulk as described above.  
\end{itemize}

\item Simulations were initialized with monoclonal populations at carrying capacity consisting of individuals having no beneficial mutations in either trait and relative fitness set zero.
\begin{itemize}
\item We allowed the simulation to run for 5,000 generations to allow the population to achieve beneficial mutation-selection balance before collecting data.
\item This ensured that the transient effects from the initializing would not distort the statistics collected for the distribution.
\item Following the burn period of the simulation, we recorded the existing classes in the bulk and their respective abundances every one 100 generations and intermittent times at which new classes established.
\item The data was used to compute the marginal mean fitness and variances of each trait, as well as their covariance.
\item From these quantities we also computed the instantaneous rate of adaptation (eqtn. ??), the mean rate of fitness increase in each trait and the overall rate of adaptation.
\end{itemize}

\end{itemize}

% figure to include in methods
\begin{figure}[h]
	\centering
	\includegraphics[width=0.35\linewidth]{2dfitnessdistributionexample.png} 
	\caption{Two-dimensional fitness distribution.} 
	\label{fig.1}
\end{figure}
	
\begin{table}[h]
	\centering
 	\begin{tabular}{ | c | c | c | c |}
    \hline
    Organism 												& Population Size 	& Selection Coefficient & Beneficial Mutation Rate \\ \hline
    \textit{E}. coli \cite{Perfeito2007}					& $10^9 - 10^{12}$ 	& $0.5 - 2 \%$  & $10^{-5} - 10^{-4}$ \\ \hline
    \textit{S}. cerevisiae \cite{desai2007speed,Levy2015}	& $10^9 - 10^{12}$ 	& $0.5 - 2 \%$  & $10^{-5} - 10^{-4}$ \\
    \hline
	\end{tabular}
	\caption{Population parameters measured in evolution experiments for key microorganisms.} 
	\label{Table 1}
\end{table}


\section*{Results}
\begin{itemize}
\item Mean trait variances and covariances over the period of the simulation
\item Fluctuations in the components of the \textbf{G}-matrix.
\item 
\end{itemize}

%figures to include in results
\begin{figure}[h]
\centering
\includegraphics[width=0.35\linewidth]{fig2_N-10p09_c1-0d01_c2-0d01_U1-1x10pn5_U2-1x10pn5_exp1}
\caption{Average rate of adaptation, variance and covariance. Parameters $N=10^9$, $s_1=s_2=0.01$, $U_1=U_2=10^{-5}$}
\label{fig.1}
\end{figure}

\begin{figure}[h]
\centering
\includegraphics[width=0.35\linewidth]{fig1_N-10p09_c1-0d01_c2-0d01_U1-1x10pn5_U2-1x10pn5_exp1}
\captionof{figure}{Variance and covariance fluctuations over time. Parameters $N=10^9$, $s_1=s_2=0.01$, $U_1=U_2=10^{-5}$}
\label{fig.2}
\end{figure}

\section*{Discussion}
\begin{itemize}
\item Negative covariance driven by LD
\item \textbf{G}-matrix instability and its implication for the long term evolution of populations
\end{itemize}

\bibliographystyle{plain}
\bibliography{biblio}

\section*{Appedices}
\begin{itemize}
\item relationship between LD and covariance
\end{itemize}


\end{document}