\documentclass[9pt,twocolumn,twoside]{gsajnl}
% Use the documentclass option 'lineno' to view line numbers
\usepackage{caption}
\usepackage{subcaption}
\usepackage[nameinlink,capitalise]{cleveref}
% \graphicspath{{figures/}}
\newcommand{\G}{\textbf{G }}
\newcommand{\M}{\textbf{M }}

\articletype{inv} % article type
% {inv} Investigation 
% {gs} Genomic Selection
% {goi} Genetics of Immunity 
% {gos} Genetics of Sex 
% {mp} Multiparental Populations

\title{Linkage disequilibrium can drive strong and unstable variances and covariances in fitness-associated traits}

\author[$\ast$,1]{Kevin Gomez}
\author[$\dagger$]{Jason Bertram}
\author[$\dagger$]{Joanna Masel}

\affil[$\ast$]{Graduate Interdisciplinary Program in Applied Mathematics, University of Arizona, and}
\affil[$\dagger$]{Department of Evolution and Ecology, University of Arizona}

\keywords{Keyword; Keyword2; Keyword3; ...}

\runningtitle{GENETICS Journal Template on Overleaf} % For use in the footer 

%% For the footnote.
%% Give the last name of the first author if only one author;
% \runningauthor{FirstAuthorLastname}
%% last names of both authors if there are two authors;
% \runningauthor{FirstAuthorLastname and SecondAuthorLastname}
%% last name of the first author followed by et al, if more than two authors.
\runningauthor{Gomez \textit{et al.}}

\begin{abstract}
Abstract.
\end{abstract}

\setboolean{displaycopyright}{true}

\begin{document}

\maketitle
\thispagestyle{firststyle}
\marginmark
\firstpagefootnote
\correspondingauthoraffiliation{Please insert the affiliation correspondence address and email for the corresponding author. The corresponding author should be marked with a `1' in the author list, as shown in the example.}
\vspace{-11pt}%

\section{Introduction}

Natural selection acts on multiple traits simultaneously. The mean trait value in a population can change either because of direct selection on trait X, or because of selection on trait Y plus a genetic correlation between X and Y \citep{lande1979quantitative,lande1983measurement}. The role of genetic correlations in the response of quantitative traits to natural selection is described by Lande's equation ($\Delta \bar{z} = \G \hspace{.05in}\beta$), where $\Delta \bar{z}$ is the expected change in the population mean of a trait $\Delta \bar{z}$, $\beta$ is the selection gradient, and the components of the \G  matrix are the additive genetic covariances between traits. Lande's equation governs the short-term evolution of correlated quantitative traits under natural selection. 

The evolutionary trajectories of traits across many generations depend on the dynamics of the \G matrix. When the \G matrix is stable, it is possible to infer past selection gradients \cite{lande1979quantitative} and forecast future trait evolution (both direction and rate) \citep{via1985genotype,arnold1992constraints,bjorklund1996importance,Schluter1996,teplitsky2011quantitative,teplitsky2014assessing}. However, it has not come into widespread use for this purpose, perhaps in part because its stability cannot be assumed a priori. Theoretical models \citep{Turelli1988,Burger1994,Jones2003,Jones2004}, comparative studies \citep{bjorklund2013rapid,Waldmann2000}, and experimental evolution \citep{wilkinson1990resistance,Shaw1995,Phillips2001} have all demonstrated that rapid change in the \G matrix is possible. How stable \G is in natural populations remains an open question \citep{Steppan2002,Arnold2008}.

Comparative studies suggest that at least certain aspects of \G might be stable \citep{Arnold2008}. Early methods for comparing \G were inconclusive about its stability, but these relied on statistical approaches that compared the magnitudes of the individual variance and covariance elements \citep{atchley1992evolutionary,lofsvold1986quantitative,Brodie1993,paulsen1996quantitative,roff1999does,Roff1999}. In contrast, more recent methods that examine common principal components of \G matrices have shown that certain structures related to their shape and orientation are preserved between closely related populations \citep{Arnold2008}. 

Stable aspects of the \G matrix have been interpreted as stable pleiotropic constraints \citep{lande1979quantitative,Arnold2008}. However, the genetic correlations described by \G are due to the combined forces of \M (mutation matrix) and selection, with much attention on \M, whose covariances derive from both pleiotropy and linkage disequilibrium \citep{Saltz2017}. A dominant role for pleiotropy was favored by the historically influential Edinburgh school of quantitative genetics, who argued that recombination would quickly eliminate linkage disequilibrium \citep[Chapter~20]{fox2006evolutionary}, particularly in the randomly mating animal populations that the Edinburgh school focused on \citep{lande1979quantitative,lande1980genetic,Arnold2008} (more).  In contrast, quantitative geneticists in the alternative, Birmingham school largely worked on inbred lines of plants, where the effects of linkage disequilibrium were impossible to ignore. The Edinburgh view came to be widely accepted, and as a result, most models for the evolution of the \G  matrix assume that genetic correlations are pleiotropic and go on to examine how evolutionary processes influence them \citep{Turelli1988}. We know far less about the about the dynamics of genetic correlations arising from linkage disequilibrium.

To study how linkage disequilibrium changes genetic variances and covariance over time, we have to consider explicit alleles or genotypes, not just quantitative traits. Beneficial mutations are discrete not infinitesimal, and appear on distinct genetic backgrounds. Selective sweeps compete with one another, and so many beneficial mutations are lost in a process known as ``clonal interference''. This will slow down adaptation \citep{hill1966effect} unless recombination brings them together, reducing the negative linkage disequilibrium between competing beneficial mutations \citep{fisher1930genetical,muller1932some}. Classical models of clonal interference considered only two loci each with two alleles. More recently, traveling wave models have been developed to describe clonal interference among large numbers of loci, arising at an arbitrary mutation rate \citep{rouzine2003solitary,desai2007beneficial,park2010speed,good2012distribution,neher2010rate,fisher2013asexual,rouzine2010multi,rouzine2007highly,rouzine2008traveling}. These models have documented that clonal interference can have an enormous impact on adaptation rates. However, to date, most of these models have treated only the evolution of ``fitness'' rather than the evolution of individual traits and their correlations. 

% * <masel@email.arizona.edu> 2018-01-24T22:40:30.992Z:
% 
% > owever, to date, these models have treated only the evolution of ``fitness'' rather than the evolution of individual traits and their correlations. 
% You mentioned one paper that might be an exception - find it!
% 
% ^ <kgomez81@math.arizona.edu> 2018-02-10T00:01:15.252Z:
% 
% I am still looking for this reference.
%
% ^.

Here, we focus on asexual populations, because they have the strongest clonal interference and resulting linkage disequilibrium. Moreover, biology today has come to recognize the numerical prevalence and ubiquitous significance of relatively asexual microbes \citep{mcfall2013animals}.

We begin with Desai and Fisher's \citep{desai2007beneficial} framework of fixed population size $N$, beneficial mutations at rate $U$, and per-beneficial-mutation selection coefficient $s$. This yields the rate of adaptation \citep[Equation 41]{desai2007beneficial}
\begin{equation}\label{eq:1}
v(U,N,s) = s^2\frac{2\log(N s)-\log(s/U)}{\log^2(s/U)}.
\end{equation}
Equation \ref{eq:1} holds when the parameters satisfy $NU \gtrsim 1/\ln(Ns)$ and $U/s \ll 1$ (the ``concurrent mutations regime''  where beneficial mutations appear rapidly relative to the time required for any one of them to fix). When combined with Fisher's fundamental theorem, Equation \eqref{eq:1} also gives the additive variance in fitness $\sigma^2$. When there is only one adaptive trait, $\sigma^2$ is that trait's additive variance. 

With two adaptive traits, Equation \eqref{eq:1} still provides overall fitness variance $\sigma^2$, but it does not give its decomposition into the variances and covariance expressed in Lande's equation. Consider just two traits, each experiencing beneficial mutations at rate $U$ which each increase fitness by $s$. Equation \ref{eq:1} indicates that the overall adaptation rate is $v(2U,N,s)$, and so by symmetry, adaptation in each trait alone occurs at rate $0.5 \hspace{.02 in} v(2U,N,s)$. This is lower than the rate $v(U,N,s)$ when the other trait is not evolving. The reduction in trait-specific adaptation rate must be distributed between the two terms of Lande's equation $v_1 =\sigma_1^2 +\sigma_{1,2}$. More importantly, the contributions of each term will over time change if the \G matrix is not constant.

Here we analyze a two-dimensional traveling wave model of an asexual population, and find that clonal interference alone, in the absence of pleiotropy, leads on average to greatly elevated variance of fitness-associated traits and to negative covariance between them.  However, the \G matrix is highly unstable over the timescale of selective sweeps.

\section{Materials and Methods}
\label{sec:materials:methods}
We consider an asexual population of haploids with fixed total population size $N$ evolving in continuous time. To keep the model simple, we assume that beneficial mutations occur with the same rate of $U$ mutations per trait per birth giving a fitness benefit $s$ with no epistasis. Deleterious mutations are not considered. Since there is no epistasis, all individual with $i$ mutations in the first trait and $j$ in the second have the same Malthusian relative fitness $r_{i,j}= i s+j s$. We refer to each such set of individuals with the same $i$ and $j$ as a ``class'' and denote their abundances and frequencies by $n_{i,j}$ and $p_{i,j}$ respectively. The population's mean fitness is denoted $\bar{r}$.

\begin{figure}[!ht]
\includegraphics[width=1\linewidth]{fig1a.pdf}
\caption{Representative two-dimensional genotype distribution. Individuals with equal numbers of beneficial mutations in each trait are combined into classes (boxes). Abundances of the bulk (solid boxes) behave deterministically, while abundance of the stochastic front (patterned boxes) behave stochastically. The red dot marks the average number of beneficial mutations in each trait, and equally fit classes are found along the red fitness isocline). In asexuals, beneficial mutations must occur on the fittest genetic backgrounds in order to contribute to the adaptive process. These mutations are represented by arrows from the bulk into the stochastic front. Classes in the stochastic front become part of the bulk once their abundances have become sufficiently large, pushing the stochastic front to include new classes. As classes below the population mean fitness isocline decline and those above increase exponentially, a two-dimensional traveling wave is produced. (Simulation parameters: $N=10^9$, $s=0.02$, and $U=10^{-5}$)}\label{fig:1}
\end{figure}

We assume that $N$ is so large that selection is strong relative to drift ($1/N \ll s$). In this case, most classes are so large that their frequency grows or declines approximately deterministically due to selection. We refer to these classes as the ``bulk''  (Figure \ref{fig:1}, solid cells). However, adaptive mutant lineages arising along the high-fitness ``front'' of the population will start with a single individual, and initially behave stochastically (Figure \ref{fig:1}, patterned cells; arrows show mutations). Many of these lineages will go extinct before attaining appreciable frequencies, but some will make it to high enough abundances that they begin to grow deterministically. This transition from stochastic to deterministic behavior is called ``establishment'' and occurs if the lineage reaches an abundance of order $1/s$. Beneficial mutations within the bulk are ignored because these are lost to clonal interference. 

The coupled dynamics of the bulk and front determine genetic variances and covariances in the population. The genetic structure of the bulk influences which new adaptive mutants will establish in the future, but  the bulk itself is the outcome of previous mutant establishments. The steady-state patterns of genetic variation are determined by beneficial mutation-selection balance: while selection in the bulk continually acts to eliminate genetic variation, beneficial mutations create new genetic variation. However, the resulting patterns of genetic variation are not simple to analyze mathematically, particularly because we are interested in the way that genetic variation is  partitioned between two traits \citep{pearce2017rapid,Desai2013}. Moreover, the fluctuations around this steady state due to the stochasticity of the front have important consequences for patterns of genetic variation, and these are even harder to analyze \citep{hallatschek2011noisy,fisher2013asexual}. Thus, we performed numerical simulations.

Our simulations follow an iterative process with two steps. First, we determine the next class to establish, which depends on the growth of classes producing adaptive mutants in the front. Once a class has established, it begins to alter the growth dynamics of the other classes in the bulk. The second step involves producing the solutions for the dynamics of the bulk with the newly established class. The process then repeats. 

To determine the next class to establish, stochastic fluctuations induced by drift and mutations must be accounted for in the growth of classes at the front. The growth of classes in the front follows a branching processes fed by mutants from the adjacent (and exponentially growing) classes in the bulk.  The next establishment event and its corresponding class  is determined by sampling each of the sojourn times of mutations in the front that are destined to establish and identifying the first to appear.   The number of beneficial mutations from a class $(i,j)$ in the bulk follows an inhomogeneous Poisson  process with a rate $\lambda_{i,j}(t)=U (1+s_{i,j}) n_{i,j}(t)$, where $s_{i,j}=(r_{i,j}-\bar{r})$ is the selective advantage of an individual in class $(i,j)$, with respect to an average individual in the population. The probability that a new mutant lineage establishes is given by, 
\begin{equation} \label{eq:2}
\pi = \frac{s_{i,j}}{ 1+s_{i,j}}
\end{equation} 
(this is derived from \cite[Equation 16]{desai2007beneficial}). The sojourn time $\tau$ for the appearance of the first mutation destined to establish can be transformed into the random variable $x=\int_0^\tau \pi \lambda_{i,j}(t) dt$ that has a standard exponential distribution. The sampling of $\tau$ is be done by generating a value for  $x$ and numerically solving for $\tau$ in the integral equation. This is repeated for each class of the bulk adjacent to the stochastic front, and the next class to establish is picked to be the one with the smallest sojourn time. Note that establishment in our simulation is solely based on the time to the first mutant establishment. In reality, an establishing class consists of multiple mutant lineages \cite{Desai2013}, but in practice this makes little quantitative difference (see \cite[p.1773]{desai2007beneficial}). 

A class that is chosen to establish is immediately incorporated into the bulk with an initial size that reflects stochastic effects on the growth of the single mutant lineage that drives its establishment. If a mutant lineage that is destined to establish appears at time $t=0$, it requires $T$ generations to grow to size $1/s_{i,j}$. The variable $T$ is random due to the effects of drift on the growth of the lineage, but its distribution is known and given by \cite[Equation 12]{desai2007beneficial}.  Once this lineage establishes $T$ generations later, the time dependent growth of its class is approximately $n_{i,j}(t) =s_{i,j}^{-1} \exp[{s_{i,j}(t-T)}]$. We consider instead an initial size  $\nu = s_{i,j}^{-1}\exp[-s_{i,j}T]$, and equivalently regard $n_{i,j}(t) = \nu s_{i,j}^{-1} \exp[{s_{i,j}t}] $ as describing the deterministic growth of the class beginning at time $t=0$. Consequently, each class that establishes can be incorporated into the bulk from the moment at which mutation driving establishment appears, but its initial size will be $\nu$. The random variable $\nu$ can be sampled by using its cumulative distribution function:
\begin{equation}\label{eq:3}
P(\nu \leq \nu_0) = 1- e^{-\pi_{fix} \nu_0},
\end{equation}
which is obtained from the probability density function of  $T$ in \cite[Equation 12]{desai2007beneficial}.

Following establishment, we update the set of classes in the bulk and calculate solutions to their time-dependent growth. Initially, low-fitness classes with less than one individual are regarded as having gone extinct and removed from the bulk, and the newly established class is added to the bulk. This will cause $N(t)=\sum n_{i,j}(t)$ to deviate slightly from $N$. To  maintain a total population size at $N$, we assume that selection in the bulk is governed by \cite[p.133]{Crow1970}
\[ \dot{n}_{i,j} = (r_{i,j}-\bar{r})n_{i,j} - \bar{r}\left (\frac{N(t)}{N}-1 \right)n_{i,j}. \] 
Since the removal and addition of rare types does not change $N$ much, this equation behaves almost identically to the standard selection equation $\dot{n}_{i,j}(t) =(r_{i,j}-\bar{r})n_{i,j}(t)$. Initial values are the size of each class at the time of establishment. This coupled system is solved one thousand generations into the future, which is then used to determine the next establishment event. 

Simulations were initialized with monoclonal populations at carrying capacity consisting of individuals having one beneficial mutation in either trait and relative fitness set to zero. We varied either $N$, $s$ and $U$ while fixing the others. Each was varied across a range constructed from values reported by evolutionary experiments involving either E. coli or S. cerevisiae \citep{desai2007speed,Levy2015,Perfeito2007}. We allowed the simulation to run for 5,000 generations, where one generation is given by $1/s$ [flag]. This allowed the population to achieve beneficial mutation-selection balance before collecting data, and ensured that the transient effects from the initializing would not distort the statistics collected for the distribution. Following the burn-in period of the simulation, we recorded the existing classes in the bulk and their respective abundances. 

HERE: The frequencies $p_{i,j}$ of the bulk are used to calculate all of the quantitative genetic parameters that characterize the evolution of traits. Classes in the front are too small to affect the means, variances and covariances of the distribution  ($1/s \ll N$).  The marginal mean fitnesses are  $\bar{r_1}=\sum_{i,j} p_{i,j} is_1$ and $\bar{r_2}=\sum_{i,j} p_{i,j} j s_2$. These  yield the mean contributions to fitness in the average individual from each trait. The genetic variances and covariance are $\sigma_1^2 = \sum_{i,j} p_{i,j} (r_i-\bar{r_1})^2$, $\sigma_2^2 = \sum_{i,j} p_{i,j} (r_j-\bar{r_2})^2$, and  $\sigma_{1,2} =\sum_{i,j} p_{i,j} (r_i-\bar{r_1})(r_j-\bar{r_2})$. According to Lande's equation, the evolution of the population can be broken down into the evolution of the two traits:
\begin{equation}\label{eq:4}
\left( \begin{array}{c}
\dot{\bar{r_1}} \\
\dot{\bar{r_2}} 
\end{array} \right)
=
\left( \begin{array}{cc}
\sigma_1^2 & \sigma_{1,2} \\
\sigma_{1,2} & \sigma_2^2 
\end{array}\right)
\left(\begin{array}{c}
1 \\
1 
\end{array}\right),
\end{equation}
 expressed. Because there is no epistasis, $\bar{r} = \bar{r}_1 +\bar{r}_2$, and the instantaneous rate of adaptation of the population is $v(2U,N,s) = \sigma_1^2 +\sigma_2^2 + 2 \sigma_{1,2}$.  

\section*{Results}
\label{sec:results}
The reduction in trait-specific adaptation $v_1$ from $v(U,N,s)$ down to $0.5 \hspace{.02 in} v(2U,N,s)$ (Figure \ref{fig:2}, extent to which solid lines are below 1), caused by clonal interference with the second trait, is driven by high levels of negative covariance (Figure \ref{fig:2}, closed circles). This is partly compensated for by a substantial increase in additive genetic variance (Figure \ref{fig:2}, open diamonds), because the negative covariance slows the removal of additive variance from the population. 

In other words, clonal interference in a rapidly adapting population can explain the common observation of high genetic variance in a fitness-associated trait combined with strong negative covariance with other fitness-associated traits, even in the complete absence of pleiotropic trade-offs. This is striking, because this kind of evidence is often interpreted as evidence for trade-offs. 

\begin{figure*}[!ht]
\centering
\includegraphics[width=1\linewidth]{fig2.pdf} 
\caption{Mean values in variance (open diamonds) of one trait and its covariance (closed circles) with the other, with their expected sum equal to the analytically calculated adaptation rate (solid line). Each average was calculated from one simulation that ran 1,000,000 generations. Variance of the focal trait is always higher than it would be in the absence of the second trait ($\sigma_1^2 >1$); negative covariance more than cancels this out ($|\sigma_{1,2}| > \sigma_1^2-1$) to reduce the adaptation rate.  Mean variance and covariance are insensitive to changes in the population size $N$ (c), but exhibit some dependence on the values of $s$ (a) and $U$ (b). For the parameter values not being varied on the x-axis, $s=0.02$, $U=10^{-5}$ and $N=10^9$. }\label{fig:2}
\end{figure*}

The reduction in $v_1$ due to the effects of clonal interference is insensitive to changes in the values of $s$ (Figure \ref{fig:2}a, solid lines),  $U$ (Figure \ref{fig:2}b), or $N$ (Figure \ref{fig:2}c). Generalizing to $k$ traits, each with equal $U$ and $s$, from Equation \eqref{eq:1}  we derive
\begin{equation}\label{eq:6}
 \tilde{v}_1 (U,s,N;k) 
= \frac{1}{k}\frac{\log^2[s/U]\log[N^2s(kU)]}{\log^2[s/(kU)]\log[N^2sU]},
\end{equation}
in units of multiples of the rate of adaptation in isolation, $v(U,N,s)$. Each additional trait increases clonal interference on the focal trait by a diminishing amount, with curvature apparent even with respect to the logarithm of the number of traits (Figure \ref{fig:3}). This suggests that much can be learned even from the simplest case of clonal interference between only two fitness-associated traits.

\begin{figure}[!ht]
\centering
\includegraphics[width=1\linewidth]{fig3.pdf}
\caption{The rate of adaptation of a focal trait as a function of the number of total traits undergoing adaptation. The decrease from each additional trait ($n>2$)is marginally less and decays to zero in the Equation \eqref{eq:6}, but in practice the expression is only valid when $kU/s \ll 1$. The effects of clonal interference on trait evolution are most significant in the case of two traits. The parameters are fixed at $s=0.02$, $U=10^{-5}$ and $N=10^9$.}
\label{fig:3}
\end{figure}

Under beneficial mutation-selection balance, the components of the \G matrix fluctuate far more than the instantaneous rate of adaptation, with transient spikes in additive genetic variance $\sigma_1^2$ mostly canceled out by transient spikes in negative covariance $\sigma_{1,2}$ (representative simulation shown in Figure \ref{fig:4}). As a result, fluctuations in variance and covariance can be substantially larger than those of total variance in fitness.

\begin{figure}[!ht]
\includegraphics[width=0.905\linewidth]{fig4.pdf}
% * <masel@email.arizona.edu> 2017-10-28T23:28:47.376Z:
% 
% > fig3.pdf
% Need tick labels on the x-axis.
% 
% ^.
\caption{Trajectories of the variance of trait one (dashed line), covariance (dotted line), and the instantaneous rate of adaptation (solid line), plotted over 10,000 generations. The mean values of each are included as lines with matching line-styles. The \G matrix over this period of time is unstable. Its two components, trait one variance and covariance, fluctuate significantly relative to the variance in total fitness over identical time periods. The trajectories were sampled from one simulation with parameters $s=0.02$, $U=10^{-5}$ and $N=10^9$. The first eigenvalue is stable relative to the second, and the coefficient of variation (COV) of $\lambda_1$ is 44.1\%, while the COV of lambda 2 is 79.3\%. Eigenvalues of the \G matrix ($\lambda_1$ and $\lambda_2$) plotted across 10,000 generations, along with the angle $\theta$ between the first eigenvector and selection gradient (solid black line). The first eigenvector defines the direction that is aligned most closely with the selection gradient $\beta$, and as a result, its eigenvalue $\lambda_1$ is smaller in magnitude due to selection degrading most genetic variance along this direction. In contrast, genetic variance along the direction of the second eigenvector $\lambda_2$ which is not under selection varies closely with covariance. Changes in the angle $\theta$ between the first eigenvector and $\beta$ depend largely on the magnitude of the covariance. Smaller covariances indicate increases the efficiency of selection in trait with most variance, and as a result, the angle $\theta$   changes to reflect which of the traits has largest variance; trait 1 if $\theta$ positive or trait 2 if $\theta$ negative. }\label{fig:4}
% * <masel@email.arizona.edu> 2017-10-28T23:32:29.075Z:
% 
% > along with variance in total fitness
% I thought you were plotting sigma_1^2+sigma_12 which is the instantaneous adaptation rate of trait 1, but here you say that you are plotting something else. Please confirm.
% 
% ^ <kgomez81@math.arizona.edu> 2018-02-11T06:41:44.720Z:
% 
% The solid line is the instantaneous rate of adaptation in fitness.
%
% ^.
\end{figure}

The two-dimensional traveling wave of Figure \ref{fig:1} can be projected onto a one-dimension traveling wave of fitness, where classes along a fitness isocline such as those along the red line in Figure \ref{fig:1}, are aggregated into fitness classes (Figure \ref{fig:4}). The shape of the one-dimensional fitness distribution is Gaussian and highly peaked near the population's mean fitness \citep{desai2007beneficial}. This means that the \G matrix will dominated by the distribution of genotypes in this fitness class (red line in Figure \ref{fig:1}, mode in (Figure \ref{fig:4}). As adaptation continues, a fitter class becomes the most abundant that will have a different distribution of genotypes and hence different trait variances and covariance. On average, the time required for this shift is the mean time between establishments at the nose of the of the traveling wave, denoted $\tau_q$ in \citet{desai2007beneficial}.
\begin{equation}\label{eq:8}
\tau_q = \frac{\log^2(s/U)}{s(2\log(Ns)-\log(s/U))}
\end{equation}

\begin{figure}[!ht]
\includegraphics[width=0.49\linewidth]{fig5a.pdf}
\includegraphics[width=0.49\linewidth]{fig5b.pdf}
\caption{The dynamics of the \G matrix components are governed by time scale $\tau_q$ of the one-dimensional traveling wave of fitness classes. Variances and covariance are dominated by the set of genotypes that make up the peak of the one-dimensional traveling wave, whose shape is approximately Gaussian. At time $t$ the peak is composed of classes shown in blue, whose frequencies determine the components of the \G  matrix. At time $t+\tau_q$ the mean shifts (relatively abruptly, due to the exponential growth of classes fitter than the mean), and the set of genotypes shown in green become the dominant group, whose distribution dominates the new trait variances and covariance.}\label{fig:5}
% * <masel@email.arizona.edu> 2017-10-28T23:57:21.626Z:
% 
% > whose shape is approximately Gaussian
% The text hasn't really explained or provided a citation for this point.
% 
% ^.
\end{figure}

Genotypes frequencies within the peak of the traveling wave are set during the stochastic growth phase of the fitness class prior to establishing \citep{Desai2013}.  As previously discussed, beneficial mutations that give rise to mutants in the nose produce distinct but equally fit mutant lineages. Many of lineages become extinct, but those that manage to survive are subsequently and equally amplified by selection following establishment. Since the relative sizes of the surviving lineages with respect to one do not change once they begin growing deterministically, the frequencies of genotypes in a fitness class become ``frozen'' at the time of establishment. Beneficial mutations that occur after that time contribute very little to the size and composition of fitness class, since their corresponding lineages will be exponentially smaller than those which initially appeared. After a fitness class has established, it eventually goes on to become the dominant group of the fitness distribution, or rather, it forms the peak of the traveling wave. The time between its establishment and its succession as the dominant fitness class is called the sweep time (denoted $\tau_{sw}$). Consequently, the frequencies of genotypes in the nose determine the values of variance and covariances approximately $\tau_{sw}$ generations later. Figure \ref{fig:6} plots the correlation between trait covariances of the bulk and nose, as a function of a time offset. The signal peaks at approximately $\bar{\tau}_{fix}$ (mean time to fixation). It shows that covariances between traits are largely set by the behavior at the nose of the one dimensional traveling wave, which consists of genotypes along the lead fitness isocline through the stochastic front.  The values of variances and covariance will be correlated from one transition to the next, because mutational influx depends on the abundances of adjacent less-fit genotypes.

\begin{figure}[!ht]
\includegraphics[width=1\linewidth]{fig6a.pdf}
\caption{Correlation between covariance of stochastic front and bulk. The correlation between the two peaks when their signals are offset by the mean sweep time of the nose $\bar{\tau}_{SW}$ (dashed line), indicating that the dynamics of the stochastic front explains 69\% of the covariance lated detected in the bulk $\bar{\tau}_{SW}$ generation after.}\label{fig:6}
% * <masel@email.arizona.edu> 2017-10-28T23:57:21.626Z:
% 
% > whose shape is approximately Gaussian
% The text hasn't really explained or provided a citation for this point.
% 
% ^.
\end{figure}
Fluctuations in the individual components $\sigma_1^2$, $\sigma_1^2$, and $\sigma_{1,2}$ are large relative to the expected rate of adaptation of each trait (Figure \ref{fig:4}), but the shape and orientation of the \G matrix are largely preserved throughout its evolution. The \G matrix expressed in Equation \ref{eq:4} for two symmetric traits ($E[\sigma_1^2] = E[ \sigma_2^2]$ and $v_1 = v_2 $ ) has eigenvalues
\[ \lambda =  \frac{\left(\sigma_1^2+ \sigma_2^2\right)}{2}\pm \frac{1}{2} \sqrt{ (\sigma_1^2 -\sigma_2^2)^2 +4 \sigma_{1,2}^2 } \approx \frac{\left(\sigma_1^2+ \sigma_2^2\right)}{2} \pm  |\sigma_{1,2}|.  \]
The first eigenvalue is the rate of adaptation in one trait, while the second is the variance in the difference between the two traits. The principle components are $\vec{x}_1 = \frac{1}{\sqrt{2}}(1,1)^T$ and $\vec{x}_2 =\frac{1}{\sqrt{2}}(1,-1)^T$ for the first and second eigenvalues, respectively.  

The total variance in fitness $\sigma^2 = \beta^T \textbf{G} \beta = \lambda_1 \langle \beta,x_1 \rangle^2+ \lambda_2 \langle \beta,x_2 \rangle^2  = \sqrt{2}[\lambda_1 + (\lambda_2 -\lambda_1)\sin^2(\phi)]$, where $\phi $ is the angle between the selection gradient and the first eigenvector. 

\section*{Discussion}

% * <masel@email.arizona.edu> 2018-01-17T17:50:06.655Z:
% 
% > Discussion}
% Include speculation about the effects of adding deleterious mutations.
% 
% ^ <masel@email.arizona.edu> 2018-01-24T22:30:26.120Z:
% 
% Compare to all other theoretical sources of G-matrix instability eg. mutation drift migration etc. and argue it is strong relative to others.
%
% ^.

\label{sec:discussion}
Although we have not included deleterious mutations in our model, their effects on the dynamics of the \G matrix will be negligible in most situations involving rapidly adapting populations \citep{desai2007beneficial,fogle2008clonal,rouzine2008traveling,good2012distribution,yu2010asymptotic}. In large populations such as ours, evolution is dominated by the influx of beneficial mutations as long as the fitness effect of deleterious mutations $s_d$ are on the order of $s$ or larger ($s_d \geq s$). Clonal interference prevents deleterious lineages from growing to any significant size that might change the behavior of two-dimensional distribution, just as it does with beneficial mutations occurring within the bulk. This will hold as long as the deleterious mutation rate isn't too large relative to the beneficial one. Still, prior work by \cite{goyal2012dynamic} shows that significantly less than a half of all mutations must be beneficial in order to ensure that adaptation. Deleterious mutations with fitness effects $s_d \leq s$ present much more subtle issues due to their ability to hitchhike, and their effects of on the adaptation of asexuals is still not relatively well understood \cite{good2014deleterious}. However, these mutations could potentially influence the evolution of the \G matrix in several important ways and should be explored in future work.  
% * <masel@email.arizona.edu> 2018-03-07T18:09:01.393Z:
% 
% > Although we have not included deleterious mutations in our model
% Start with discussing what background selection is, how it fits into the paper, why it might matter, before arguing that it doesn't.
% 
% ^.

The effects of selection, drift, mutation and recombination on the evolution of \G is given by the expression:
\[\Delta G = G(\gamma - \beta \cdot \beta^T)G + 2 M + 2\sum_{i,j=1}^n r_{i,j}(C_{i,j}-C_{i,j}'),\]
given in \cite[Chapter 20, Box 3]{fox2006evolutionary}. The first term includes directional selection ($\beta$ terms) and the effects of stabilizing and correlational selection which reorienting the G matrix ($\gamma$ term). The second term involves the mutational covariance matrix \textbf{M}, and gives the sum of mutational covariances generated by the effects of pleiotropic mutations at each locus. The third term gives the rate of degradation in covariance caused by linkage disequilibrium, which depends on the recombination rate. More terms may be needed depending on whether the distribution of allelic effects are normal (Barton and Turelli 1987). More detailed explanations of each terms are given in  \cite[Chp34,35,36]{lynch2018evolution}, but the last term is discussed under assumptions that are valid for infinitesimal model.

% (Overview of Genetic Drift and its relevance in our model)
With respect to the importance of mutations, pleiotropy is largely emphasized \cite{reeve2000predicting,Steppan2002,Arnold2008}. Changes in genetic correlaitons due to linkage disequilibria enter into the dynamics of \G through the mutation matrix \textbf{M}. 
\cite{Phillips2001}
\cite{lande1979quantitative} 

% (Fluctuations in G due to Mutations)
\cite{Arnold2008} has a section on stability of the G matrix when mutational effects evolve (Jones papers cited). 
\cite{Jones2003} The nature of mutations at pleiotropic loci can dramatically influence stability of G, i.e. when a mutation at a single locus simultaneously changes the value of the two traits (due to pleiotropy) and these effects are correlated, mutation can generate extreme stability of G. 
\cite{Jones2007} examine a two trait system and find that mutational correlation does evolve in response to selection on the bivariate phenotype (mutation is third quant-trait).

\cite{Johnson2005} has a discussion of the effects of mutation on quantitative genetic variation that should be discussed here. 

% Quantitative genetic models for the dynamics of \G attribute changes in size are thought to be driven by drift \cite{lande1979quantitative,Roff2000,Arnold2008}. In our model, changes in the size of the \G matrix are induced by stochastic fluctuations in the front resulting from the effects of drift.  
% (Fluctations in G: Nose Mutations vs Mutation Matrix)
% \cite{Jones2007} 
% \cite{Arnold2008}
% \cite{HANSEN2008}

% (Genetic correlations as constraints to trait evolution)
% \cite{hine2014evolutionary} 

% (Comparison with Barton and Otto's Negative Linkage Disequilibria Results, and other related quantitative genetics)
% \cite{Barton2005} 
% \cite{Bulmer1971} Selection induces changes in variance contributions from linkage disequilbrium are shaped by selection, since the infinitesimal model, frequencies 
% \cite{}
% In our model, covariance between traits is arises entirely from linkage disequilibrium. We can assume that specific alleles at either locus correspond to the number of beneficial mutations that have accumulated in the corresponding trait.  It follows that the $p_{i,j}$ are the gametic frequencies for the genome consisting of allele $i$ paired with allele $j$, and $D_{i,j} = p_{i,j}-p_{i\cdot} p_{\cdot j}$ is their coefficient of linkage disequilibrium ( $p_{i\cdot}$ and $p_{\cdot j}$ are marginal distributions). By applying \textbf{equation 2}, we can derive a mathematical relationship between covariance and linkage disequilibrium (need to cite chapter 5, Walsh and Lynch).
% % * <masel@email.arizona.edu> 2017-07-29T17:34:47.794Z:
% % 
% % > specific alleles at either locus correspond to the number of beneficial mutations that have accumulated in the corresponding trait
% % Multiple loci can contribute to the same trait. An allele cannot be a "number of beneficial mutations", because an allele must pertain to one locus. Stick to the word "genotype" not "allele", eg "genotype i,j has i beneficial mutations in trait 1 and j beneficial mutations in trait 2 compared to the ancestral population". This means that you cannot calculate linkage disequilibrium as you do: linkage disequilibrium is generally defined as being between two particular beneficial mutations. What you calculated was simply a covariance. Because of this confusion, I got a bit lost here and so skipped Equation 6.
% % 
% % ^.
% \begin{equation}
% \sigma_{1,2}=\sum_{i,j}D_{i,j}[(i-\bar{i})s_1][(j-\bar{j})s_2]. 
% \end{equation} 
% As the population adaptsAcknowledgments, mutations and selection  increase and decrease the amounts LD which ultimately determine how traits interact in their evolution.

% We examined the resulting dynamics of trait evolution using numerical simulations of our model using Mathematica. We focused on modeling the symmetric case, in which the two traits had identical mutation rate and selection coefficients. These, along with the population size, were chosen to match values reported in evolutionary experiments involving relevant micro-organisms (see Table 1). 

% The behavior of the stochastic front is an evolving boundary in two-dimensional trait space. \cite{pearce2017rapid} derive features of the steady state two-dimensional front in the asymptotic limit $\log(N s) \rightarrow \infty$, but these do not characterize its stochasticity. As we will later show, the dynamics variances and covariance depend significantly on these fluctuations. Thus, we proceed by simulation. [Polish]

\section*{Acknowledgments}
\label{sec:acknowledgments}
We thank Patrick Phillips and Bruce Walsh for helpful discussions. Funding was provided by the National Science Foundation (DEB-1348262) and the National Institutes of Health (T32 GM084905).
\bibliography{biblio}

\section*{Appendix}
The derivation of the expression given in \ref{eq:4} from Lande's equation is as follows. The derivative measuring the change in the mean of a trait $\dot{\bar{r_1}} \approx \Delta \bar{r_1}/\Delta t$ for a sufficiently small interval of time. Lande's original expression yields,
\[
\left[
\begin{array}{c}
\Delta \bar{r_1} \\
\Delta \bar{r_2} 
\end{array}
\right]
=
\left[
\begin{array}{cc}
\sigma_1^2 & \sigma_{1,2} \\
\sigma_{1,2} & \sigma_2^2 
\end{array}
\right]
\left[
\begin{array}{c}
\beta_1 \\
\beta_2 
\end{array}
\right]
\]
The selection gradient on the right hand side is the change in fitness with respect to change in the trait value, indicating that $\beta_1=\beta_2=s$. Dividing both sides by $\Delta t$ allows us to equate the left side to the column vector $(\dot{\bar{r_1}},\dot{\bar{r_2}})^T$, while the scaled selection gradient becomes $(s/\Delta t,s/\Delta t)$. Setting $\Delta t = s$ provides equation 5.   

\section{Relevant That Needs Discussion}
in asexuals; this fact has, for example, been the basis of explanations for the evolutionary advantages of sex and recombination \citep{Barton2005,Otto2009}

Environmental change affects genotype $\times$ environment interactions and hence changes the \G matrix stability, so to assume a constant \G matrix, one must assume the absence of GxE interactions. Simulations have revealed that large population sizes and strong correlational selection can induce stability in \G's dynamics \cite{Jones2003}. When mutational covariances can respond to selection, the \M matrix and hence the \G matrix is more stable \cite{Jones2007}. 


\end{document}Acknowledgments