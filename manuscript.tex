\documentclass{article}
\usepackage[margin=0.5in]{geometry}
\usepackage{amsmath}
\usepackage{amssymb}
\usepackage{natbib}

\begin{document}
\title{The Speed of Evolution in a Two Dimensional Trait Space}
\author{Kevin Gomez, Joanna Masel, Jason Bertram}
\date{2017}

\maketitle
\newpage

\begin{abstract}
This paper will discuss something.
\end{abstract}

\section*{Introduction}

%-----------------------------------------------------------------------------------------
\section*{Methods}
\subsection*{A Two Trait Model with Selection and Mutations}
The construction of our model employs much of the theoretical framework developed by Desai and Fisher \citep{DesFish07}, but with the key distinction of including two traits that evolve through selective and mutation forces.  We consider asexual populations with no recombination.  We assume that the population size $N$ large but fixed.  Beneficial mutations occur at a rate $U_k$, where $k=1,2$ and designates the trait in question.  A beneficial mutation modifying trait $k$ provides a fixed incremental gain $\Delta c_k$, in the relative growth rate $r_{i,j}$ of a mutant, where $r_{i,j}=1+\Delta c_1 i + \Delta c_2 j$.  We adopt the convention of using an indices $i$ and $j$ to indicate the number of beneficial mutations in the first and second trait, respectively.  This allows us to represent the fitness of an individual, or its fitness class, by a the point $(i,j) \in \mathbb{N}^2$.  The discrete fitness distribution of the population in the two dimensional trait space is then given by the set of frequencies $\{p_{i,j}\}_{i,j}$.  Relative fitness is defined as its growth rate in excess of the population's mean growth rate $r_{i,j}-\bar{r}$, .  The mean fitness of a population can also be written as $\bar{r}=1+\Delta c_1 \bar{i} + \Delta c_2 \bar{j}$, where the means are marginal with respect to the joint distribution $p_{i,j}$.  

The frequency of a fitness classes which grows deterministically is denoted as $p_{i,j}$, and its abundance is governed by the coupled equation defined in (1).
\begin{equation}
\frac{dp_{i,j}}{dt}=(r_{i,j}-\bar{r})p_{i,j}
\end{equation}
From this equation it follows that the relative fitness of an individual from the fitness class $(i,j)$ equals $d(\ln(p_{i,j}))/dt$.  


\subsection*{Mutation-Selection Balance in a Two Dimensional Trait Space}


\section*{Results}

\section*{Discussion}

\section*{Appendices}
The following notes provide some basic calculations and formulas needed for constructing and running the two dimensional model.


\bibliographystyle{plain}
\bibliography{biblio}

\end{document}