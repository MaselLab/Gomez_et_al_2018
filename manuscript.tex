\documentclass[11pt,one column]{article}
\usepackage[margin=0.5in]{geometry}
\usepackage{amsmath}
\usepackage{amssymb}
\usepackage{natbib}
\usepackage{graphicx}
\usepackage{capt-of}
\usepackage{multicol}

\graphicspath{{figures/}}

\begin{document}
\begin{multicols}{2}

\title{Modeling How Evolution in One Trait is Affected by Adaptation in Another}
\author{Kevin Gomez, Joanna Masel, Jason Bertram}
\date{2017}
\maketitle

\begin{abstract}
This paper will discuss something.
\end{abstract}

\section*{Introduction}
Natural selection frequently acts on multiple traits simultaneously. A single trait’s response to selection leads to a shift in the population mean towards the fittest phenotype.  The response to selection with multiple traits involves both direct and indirect effects that together determine their evolution \cite{Scarcelli23102007,Lovell2013,Wagner2011}.  Indirect effects arise from genetic mechanisms that underlie trait expression which may be associated or shared in a manner that enforces specific patterns of co-expression and give rise to trait correlations.\par

Trait correlations are statistical measures of associated co-expression in traits primarily used in quantitative genetics.  Within the 



Evolution with multiple traits can produce much more complex responses in populations as they adapt; a fact that was clear to Darwin (Darwin p. 144-150).  Today, evolutionary biologists are in agreement that these indirect influences stemming from simultaneous trait evolution have important consequences that affect populations on a micro- and macro-evolutionary scale.  Plant and animal breeders have been keen on this fact for much longer, and not surprisingly, our most comprehensive theory of adaptation in complex organisms with multiple traits was developed from the quantitative tools used in by them.  The extension of quantitative genetics to multivariate phenotypes, formulated by Lande (Lande 1979; 1980), provides the first characterization of selection on multiple traits, formally expressed in the breeder’s equation ($\Delta z ̅=G \beta$). The phenotypes of quantitative traits change ($\Delta z$) in response to a selection gradient ($\beta$) acting on them, given a specific set of trait variances and covariance between them ($G$). Lande’s formulation is largely independent of the principles of heredity.  However, genetic correlations are known to be the produce of either pleiotropy, linkage disequilibria, and trans acting elements.\par

In practice, geneticists can estimate $\beta$ and $G$ to determine the short-term response of quantitative traits to selection \cite{lynch1998genetics}, without knowledge of the underlying causes of that create correlations between traits.  However, most quantitative geneticist adhere to two schools of thought on this matter; these are known as the Birmingham and Edinburgh schools.  The differences in their views stem from the types of organism they studied and the types of questions they aimed to answer.\par

Quantitative geneticists in the Edinburgh school have traditionally worked on questions concerning animal breeding, and much of their statistical models aim to clarifying the process by which phenotypic quantitative traits change in response to artificial selection in animals.  Autho
In such populations, the genetic foundation of phenotypes is hidden, and random breeding are inevitably core elements that must be accounted for in their genetic analysis.  Consequently, arise from associations between alleles that form through either pleiotropy, linkage disequilibrium, or trans acting effects.\par

There are two schools of thought among quantitative geneticists that differ in the principle cause of associations between alleles.  One of them holds pleiotropy as the largely responsible for  pleiotropic effects each of them The views held by each are largely the result of distinct questions and applications Geneticist in the Edinburgh school hold that genetic correlations are the product of pleiotropic effects, while those in the Birmingham school believe that linkage The majority of quantitative genetics follow Geneticist pleiotropy.\par

We focus on asexual micro-organisms that lack recombination and examine how the evolution of their traits is affected by linkage. A variety of organisms fall under this category, which includes many single celled Eukaryotes and a variety of bacteria (cite work). In this setting, linkage disequilibria is entirely determined by the combined action of mutations, selection and drift. Standard results in population genetics demonstrate that the amount of linkage disequilibria between various alleles will depend on the beneficial mutation rate ($U_b$), the selection coefficient ($s$), and strength of drift measured by population’s size ($N$). Logically, one expects variances and covariances arising from linkage to also depend on the population’s parameters, but a general expression that details this explicit relationship has not been uncovered. The majority of results concerning linkage comes from experiments with inbred lines in plants, stemming from the Birmingham school, and that have limited application to populations of interest to us Jinks and Mather (1981).\par

In the case of adaptive traits, with fitness conferred taken as their quantitative measure, the relationship between variance and the population parameters is known explicitly. The work of Desai and Fisher (2007) provide the variance in fitness explicitly shown to work of gives the variance of adaptive traits in terms of the population’s parameters $N$, $U_b$, and $s$. The variance of traits and the covariance between them are estimated, and assume that the underlying traits satisfy basic assumptions, such as being polygenic. The variance in fitness was not something that could be determine from populations parameters, until only recently. Desai and Fisher’s work (2007) provides the rate of adaptation of the population in terms of its parameters,
\begin{equation} 
v =s^2  \frac{2 \ln(Ns)-\ln(s/U_b)}{\ln^2(s/U_b)}
\end{equation}  
and since the rate of adaptation is the variance of fitness ($v=\sigma^2$).  If we consider a second adaptive trait evolving concurrently, with identical mutation rate and selection coefficient, then the rate of adaptation in each trait should be half of the total rate of adaptation $v(2U_b)$ since evolution in each contributes equally the overall adaptation of the population. A simple application of Desai and Fisher’s formula quickly shows that the rate of adaptation for a single trait is reduced when a second trait is considered since
\[ 1/2  v(2U)<v(U) \]
Naturally, we should expect such a result since we have doubled the mutation rate, and consequently, increased the amount of clonal interference that occurs within the population. However, the mere fact that we are also considering a second trait indicates that there to this picture than this.\par

The rate of adaptation in one trait is determined by its variance and covariance with other traits, and this relation is expressed in the breeder’s equation. In our situation, both quantitative traits are adaptive and we measure them by the fitness they endow the organism. Consequently, the rate of fitness increase of the focal trait $v_1$ must be related
\begin{equation}
v_1=\sigma_1^2+\sigma_{1,2}
\end{equation}

\section*{Results}
Discuss results here
{\centering
\includegraphics[width=1.0\linewidth]{fig2_N-10p09_c1-0d01_c2-0d01_U1-1x10pn5_U2-1x10pn5_exp1}
\captionof{figure}{Mean rates of adaptation, variance and covariance.} \label{fig.1}}

{\centering
\includegraphics[width=1.0\linewidth]{fig3_N-10p09_c1-0d01_c2-0d01_U1-1x10pn5_U2-1x10pn5_exp1}
\captionof{figure}{Variance and covariance fluctuations over time.}
\label{fig.2}}

\section*{Discussion}
\section*{Methods}
We consider a population with fixed size N, in which individuals have two traits that contribute to its fitness, its growth rate, and which evolve through beneficial mutations; deleterious mutations are not considered.  The beneficial mutation rate for each trait is denoted $U_k$ (trait $k=1,2$), and each increases trait $k$’s contribution to relative fitness by $s_k$.  We can express an individual fitness as 
\[ r_{ij}=r+i s_1+j s_2 \]
where i and j represent the number of beneficial mutations in trait one and two, while $r$ is the growth rate a common ancestor to the population. 
An individual’s relative fitness is given by
\[ (r_{ij}-r ̅)=(i-\bar{i} ) s_1+(j-\bar{j} ) s_2 \]
respectively, and i ̅ and j ̅ are the population mean.  The two traits of individuals are treated as independently evolving loci in which beneficial mutations occur at a fixed rate $U_k$ ($k=1,2$).  Each beneficial mutation is either lost to drift, or the subpopulation carrying it will grow sufficiently large to ensure that the beneficial mutation fixes in the population through selection. The probability of the latter event, denoted $\pi_{fix}$, is proportional to the selective advantage of the mutation. In these cases a beneficial mutation is said to have established in the population. The selective advantage of an individual is measured by its relative fitness, which in our model, is its expected growth rate minus the growth rate of an average individual. Each new beneficial mutation in trait $k$ increases this fitness by $s_k$; we do not consider deleterious and neutral mutations in our model. If an individual has i mutations in the first trait and $j$ in the second, while the average number carried in the two traits is $\bar{i}$ and $\bar{j}$, its relative fitness equals the difference $(r_{ij}-\bar{r})=(i-\bar{i}) s_1+(j-\bar{j}) s_2$. We group members of the population that carry the same number of beneficial mutations in each trait, $i$ mutations in the first and $j$ in the second, into fitness classes that will often be denoted by subscripts $ij$. The set of fitness class abundances $\{n_{ij} \}$ induce a distribution over the trait space, represented by $N^2$, containing all fitness classes. This two-dimensional distribution, which we will often denote by the set of frequencies $\{p_{ij}\}$, travels in the direction of increasing fitness as the population adapts (Fig 1) in an analogous fashion to the motion of Desai and Fisher’s (2007) one dimensional traveling wave. We also examine the properties of the marginal distributions $\{p_{i\cdot}\}$ and $\{p_{\cdot j}\}$ for each trait, whose behaviors over time are significantly affected by genetic correlations that arise from the confluences of mutations, selection and drift.
\begin{equation}
p_{i \cdot }= \sum_j p_{ij} \hspace{0.5in} p_{\cdot j}= \sum_i p_{ij}
\end{equation}
In order to study how genetic correlations arise and influence the evolution of traits in adapting populations, we focus on regimes of adaptation characterized by significant levels of genetic variation in each trait. The results for one-dimensional traveling waves provides the relationship between variance in fitness from one trait evolving and population parameters $N$, $U_k$, and $s_k$. When mutations appear infrequently relative to the average time required for beneficial mutation to fix, we obtain the simplest behavior possible in which beneficial mutations establish in succession and spread to the population long before the next one appears ($N U_k \ll 1/\ln(N s_k)$). These populations rarely have any variance in fitness stemming from trait diversity, and are not the focus of our investigations. We focus instead on the case where many beneficial mutations establish prior to any one fixing ($N U_k \geq 1/\ln(N s_k)$), where the effects of clonal interference cause variation in fitness which persists over time in the type of populations we consider. For large populations with strong selection in each trait ($N^{-1}\ll s_k \ll 1$), changes in the abundances of fitness class are either dominated by stochastic fluctuations due to drift ($n_{ij}<\pi_{fix}^{-1}$), or growth deterministically otherwise due to selection. We will refer to fitness classes in the first group as the front of the wave, and their growth can be modeled as branching processes that are fed new mutants from adjacent less fit classes. The fitness classes in the stochastic front will consist of those which are fitter and adjacent to ones that have established whenever $s_k/U_k \gg 1$. During the stochastic phase of growth, fitness classes are much too small to influence dynamics that are driven by selection, so that they may be ignored until the moment they do establish. As for the one-dimensional traveling wave, the key variable of interest is the time-between establishments τ of new beneficial mutations. This random variable summarizes all of the stochastic effects that accumulate over time during the growth of a fitness class, and determines the rate at which fitness classes in the stochastic front transition into the deterministic regime where selection is the prominent force. We will be interested in the rate of these transitions, because they will determine two very important aspects of the two-dimensional distribution’s evolution that influence trait evolution. The abundances of fitness classes that have established, whose behavior is dominated by the action of selection, deterministically grow and decline in size. We refer to the fitness classes in this group as the bulk of the population. The quantities i ̅ and j ̅ are marginal means computed from the marginal distributions (Eq. 1).
\begin{equation}
\bar{i}=∑_i i p_{i \cdot} \hspace{0.5in} \bar{j} = ∑_j p_{⋅j} j
\end{equation}
Relative fitness, written as the difference $(r_{ij}-\bar{r})=(i-\bar{i}) s_1+(j-\bar{j}) s_2$, determines the growth and decline in abundances $n_{ij}$ that follow the differential equation
\begin{equation}
\dot{n}_{ij}=n_{ij} (r_{ij}-\bar{r})-n_{ij} \left(\frac{N(t)-N}{N}\right) r 
\end{equation}
where $N(t)=∑_{ij} n_{ij}(t)$ above, and the right-most term is the density-regulation component of growth. Stochastic fluctuations in the total population size are negligible ($|N(t)-N| \ll N$), and so we may also consider the frequencies of fitness classes in the bulk to obey Eq. 3 below.
\begin{equation}
\dot{p}_{ij}=p_{ij} (r_{ij}-\bar{r})
\end{equation}
From Eq. 3, the rate of adaptation can be expressed as the sum of the individual rates of adaptation in each trait, indicated by subscripts, as $\dot{\bar{r}}=\dot{\bar{r}}_1 +\dot{\bar{r}}_2$.
Each of the two individual rates are related to the variance in fitness from the respective trait and its covariance in fitness with other (Eq. 4).
\begin{equation}
\dot{\bar{r}}_1=\sigma_1^2+\sigma_{1,2}\\
\dot{\bar{r}}_2=\sigma_2^2+\sigma_{1,2}
\end{equation}
Mutation-selection balance induces steady state values for the speed of evolution in the two, from which it possible to determine the two variances, the single covariance, and furthermore, the resulting G-matrix of the two traits (Eq. 5).
\[
G= \left(
\begin{array}{cc}
\sigma_1^2& \sigma_{1,2}\\
\sigma_{1,2} & \sigma_2^2
\end{array}
\right) \]
Linkage disequilibria continually arises throughout the adaptation process and limits the response of traits to selection. In the case of two alleles, linkage disequilibria is a measure of the statistical association between them. For simplicity, we can regard all ith beneficial mutations occurring in trait one as the same allele, and likewise with the jth mutation in trait two. This allows us to treat the difference $D_{ij}=p_{ij}-p_{i\cdot} p_{\cdot j}$ as a measure of linkage disequilibrium between the two alleles corresponding to the $i^{th}$ mutation in trait one, and the $j^{th}$ in trait two. In turn, the covariance in fitness can be written out as Eq. 6.
\begin{equation}
\sigma_{1,2}=\sum_{ij}D_{ij}(i-\bar{i})s_1 (j-\bar{j}) s_2
\end{equation}
and it becomes evident that linkage disequilibrium directly affects the evolution of the two traits. We can also note from this expression that making a distinction between alleles that yields the same fitness advantage does not change the covariance.

Discuss the details of the simulations.

\bibliographystyle{plain}
\bibliography{biblio}
\end{multicols}

\newpage

\section*{Appendix A:}

\section*{scrap text}
 for example,  the rate of adaptation, and developed his now famous geometric model to infer that complexity would lead to slower adaptation in living organisms.  Other majors works addressing evolutionary biology would follow, such as (cite), but quantitative genetics to provide some of the most The model provides a simple geometric argument demonstrating how adaptive evolution would be constrained in higher dimensional trait spaces.  Kimura (1983) gave further extensions to this model by incorporating probability of fixation, and Orr (1998, 2000) provided additional insights on the the optimal dimensions that ensured maximum adaptation.  Some of the earliest works that attempted to address multi-trait adaptation were developed by Fisher (1930)However, early evolutionary biology could provide little development in this area, in part due to standing questions concerning the relationships between traits and their genetic foundation, such as how to reconcile Mendelian genetics and with the inheritance of continuous characters. Thus questions concerning whole organisms, which are of main interest to plant and animal breeders, and the level of whole organisms.\par

\end{document}