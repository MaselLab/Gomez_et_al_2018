\documentclass[9pt,twocolumn,twoside]{gsajnl}
% Use the documentclass option 'lineno' to view line numbers

\articletype{inv} % article type
% {inv} Investigation 
% {gs} Genomic Selection
% {goi} Genetics of Immunity 
% {gos} Genetics of Sex 
% {mp} Multiparental Populations

\title{Modeling How Evolution in One Trait is Affected by Adaptation in Another}

\author[$\ast$,1]{Kevin Gomez}
\author[$\dagger$]{Jason Bertram}
\author[$\dagger$]{Joanna Masel}

\affil[$\ast$]{Department of Applied Mathematics, University of Arizona, and}
\affil[$\dagger$]{Department of Evolution and Ecology, Univeristy of Arizona}

\keywords{Keyword; Keyword2; Keyword3; ...}

\runningtitle{GENETICS Journal Template on Overleaf} % For use in the footer 

%% For the footnote.
%% Give the last name of the first author if only one author;
% \runningauthor{FirstAuthorLastname}
%% last names of both authors if there are two authors;
% \runningauthor{FirstAuthorLastname and SecondAuthorLastname}
%% last name of the first author followed by et al, if more than two authors.
\runningauthor{Gomez \textit{et al.}}

\begin{abstract}
Abstract.
\end{abstract}

\setboolean{displaycopyright}{true}

\begin{document}

\maketitle
\thispagestyle{firststyle}
\marginmark
\firstpagefootnote
\correspondingauthoraffiliation{Please insert the affiliation correspondence address and email for the corresponding author. The corresponding author should be marked with a `1' in the author list, as shown in the example.}
\vspace{-11pt}%

\section{Introduction}

Natural selection frequently acts on multiple traits simultaneously. A single trait's response to selection leads to a shift in the population mean towards the fittest phenotype. With multiple traits, the response involves both direct and indirect effects from selection that together determine the evolution each \citep{lande1983measurement,Scarcelli23102007,Lovell2013,Wagner2011}. Direct effects are those induced by selection favoring fitter phenotypic values in each trait.  However, the traits themselves may be associated at the genetic level to an extent that only certain combinations of their phenotypic values are possible, and selection on any one of them will necessarily lead to changes in the others. For example, we can consider two traits that are controlled by a single pleiotropic locus, but with only the first under selection. Alleles frequencies at the locus change in response to selection against the first trait, and they also induce a shift in the mean of the second. These are indirect effects experienced by the second trait, which are caused by selection on the first.  Associations such as these can alter fundamental features of adaptation and have significant evolutionary consequences \citep{Felsenstein1979, Arnold2001, Arnold2008}.\par
% * <masel@email.arizona.edu> 2017-07-28T22:53:26.215Z:
% 
% > trait expression,
% "Expression" is a funny word to use. For most biologists, it means gene expression. By calling it "trait expression", you are signalling that you mean something else, but I think readers might still find it confusing. The term "expression" pre-supposes that the genetic basis for something is there, and focuses on whether it is somehow realized or not; I don't think that is what you mean.
% 
% ^.
% * <masel@email.arizona.edu> 2017-07-28T22:52:47.005Z:
% 
% > the response involves both direct and indirect effects
% You need to define what you mean by direct and indirect effects: the reader doesn't know.
% 
% ^ <kgomez81@math.arizona.edu> 2017-08-07T14:10:25.416Z.

For quantitative traits, genetic associations and their effects on adaptation are mathematically described by Lande's equation ($\Delta \bar{z} = \textbf{G} \hspace{.05in}\beta$). It states that the change in the population mean $\Delta \bar{z}$ equals the product of the selection gradient $\beta$ and what is known as the \textbf{G}-matrix, whose components are the additive genetic covariances between traits. Lande’s result has been used to extend much of our understanding of how phenotypes evolve under selection, yielding a rich set of ideas that emphasize the role of the \textbf{G}-matrix in directing adaptive evolution at both micro- and macro-evolutionary scales.\par
% * <masel@email.arizona.edu> 2017-07-29T12:43:16.916Z:
% 
% > When combined with the notions of adaptive landscapes
% The link to adaptive landscapes is not obvious, including to me. G-matrix theory is all continuous, and the only possible landscape is a phenotype-fitness map. The adaptive landscape you are going to deal with is a genotype-fitness map, ie a completely different object.
% 
% ^ <kgomez81@math.arizona.edu> 2017-08-07T14:10:30.176Z.

The behavior of the \textbf{G}-matrix over time largely depends on what are the causes of the genetic correlations. Correlations between quantitative traits can arise for a number of reasons, but they often have a genetic basis \citep{Saltz2017}. They arise from either pleiotropy, linkage disequilibria, or are the product of trans acting elements; the first two that are thought to be the most significant factors. Historically, there have been two schools of thought what the principle causes of genetic correlations are. The first was put forth by the Edinburgh school, which held that pleiotropy was principle source, since linkage disequilibrium could rapidly decay by recombination (\citep{fox2006evolutionary}, chp 20). This belief was largely motivated by interests the relationships between parent-offspring phenotypes within randomly mating populations, and its implications for evolution. The alternative view came out of the Birmingham school, which largely experimented with inbred lines of plants. Geneticists in this group believed that linkage disequilibria was much more important; a fact that often reflected in their experiments. Presently, the Edinburgh view is more widely accepted, and as a result, most models for the evolution of the \textbf{G}-matrix assume genetic correlations are pleiotropic and proceed with examining how evolutionary processes influence them. We know far less about the about the behavior genetic correlations from linkage disequilibria over time, but this is not much of an issue for most complex organisms.\par

In the case of asexuals, the Birmingham view is more appropriate, and determining how the genetic correlations due to LD change over time is critical to our understanding of adaptation. Linkage disequilibria persists for much longer periods of time in a asexuals, which can adversely affect adaptation. This fact has been used to argue for the evolutionary advantages of sex and recombination \citep{Barton2005,Otto2009}. Consequently, we can learn a substantial amount about the evolution of their traits by examining how genetic correlations from LD evolve over time and influence adaptation. Do so would not only reveal a great deal about the oldest and most abundant forms of life on the planet, such as bacteria, but could prove useful to our use of them in evolutionary experiments.\par

Our best models for adaptation in asexuals make use of what are known as traveling waves, and are based on population genetics. They have been used to establish important relationships between the population's parameter and key features of evolution in asexuals. Among these is the model developed by \cite{desai2007beneficial}, which they used to derive an expression for the rate of adaptation, given by $v(N,s,U) = 2s^2 \ln(N s)/\ln^2(s/U)$. Here, $N$ is the population size, $s$ is the effective selective advantage of a beneficial mutation, and $U$ is the beneficial mutation rate. This is an important result, because it directly provides the additive variance in fitness by applying Fisher's fundamental theorem ($v=\sigma^2$). If we regard the population as having one adaptive trait, then this also equals the trait's additive variance and realize that we have said something more about the quantitative genetics involved.\par

Unfortunately, if we apply Desai and Fisher's result to the case of two adaptive traits, we still can still solve for the overall fitness variance $\sigma^2$ of the population, but nothing can be said about the variances and covariance of the two traits. This highlights the shortcomings of the traveling wave model in the case of multiple traits, for which we should expect Lande's equation to hold. As an example, consider a single adaptive trait evolving with parameters $N$, $s$ and $U$, then its variance and rate of adaption are $v(U;N,s)$ according to the formula given above. If the population acquires a second adaptive trait with identical parameters $s$ and $U$, then the new rate of adaptation is clearly $v(2U;N,s)$. By symmetry we know that rate of adaptation in each trait alone must be $0.5 \hspace{.02 in} v(2U;N,s)$, and furthermore, the rate of adaptation in trait one is now less with the second trait. Lande's result implies that trait one's rate of adaptation, when evolving with the second, must equal the sum of its variances and covariance ($v_1 =\sigma_1^2 +\sigma_{12}$). Clearly, adding the second trait produces a new covariance term that affects $v_1$, but it also induces a change in the variance of the first trait. What we do not know is the amount that each of these contribute to the observed reduction in trait one's rate of adaptation.\par
% * <masel@email.arizona.edu> 2017-07-29T13:00:14.234Z:
% 
% > Clearly, we lack any information about how trait two is affecting trait one's evolution,
% Spell out what exactly it is that we don't know, ie the degree to which the reduction in adaptation rate of trait 1 is due to a change in variance or a change in covariance.
% 
% 
% ^.

To examine how adaptation in one trait is affected by evolution in another, we extend the work of Desai and Fisher's traveling wave model and consider a two dimensional trait space. We examine the simplest case where individuals have two non-pleiotropic traits that contribute equally to fitness. We assume that beneficial mutations in each trait occur at identical rates, affect each trait independently, and exhibit no epistatic interactions between them.  Lastly, we will focus on a population whose size remains fixed. Our aim is to explore the effects of genetic correlations arising from linkage disequilibrium contribute to the interactions that direct trait evolution. So, we assume that the two traits are controlled separate non-pleiotropic loci. Our exploration of the question will be done using simulations, whose results we present after discussing key details of the model in the next section. We will conclude with a discussion of the our findings and their implications.
% * <masel@email.arizona.edu> 2017-07-29T13:01:43.541Z:
% 
% > two independently evolving traits
% They are not evolving independently. What you mean instead is that there is no pleiotropy. You might also summarize other key simplifications here, eg constant N, no epistasis etc.
% 
% ^.

\section{Materials and Methods}
\label{sec:materials:methods}

We considered an asexual population at carrying capacity $N$ in which haploid individuals had two traits contributing to their relative fitness. Each trait was assumed to be under the control of a single locus. We assumed that beneficial mutations improving each trait appeared at a rate of $U_k$ ($k=1,2$) and increase th fitness by $s_k$; deleterious mutations were excluded. Individuals that had accumulated $i$ mutations in the first and $j$ in the second were taken to have an overall fitness of $r_{ij} = i s_1+j s_2$.  Their relative fitness is given by the difference $(r_{ij}-\bar{r})$, where  $\bar{r} $  is the population mean fitness.  

In analogous approach to that of \cite{desai2007beneficial}, we divided our population into classes bases on the number of beneficial mutations in each trait. Their abundances are accordingly denoted $n_{ij}$ and their frequencies by $p_{ij}$. We assumed that $1/N \ll s \ll 1$ to ensure that the set of classes could be neatly separated into two groups based on whether their behavior was deterministic and stochastic. Following the convention of Desai and Fisher, we refer to the set of deterministically behaving classes as the bulk, while those that grow stochastically and have a positive selective advantage ($r_{ij}>\bar{r}$) will be referred to as the stochastic front (see \textbf{fig.1}).\par

\begin{figure}
\includegraphics[width=1\linewidth]{fig1.pdf}
\label{Figure 1.}
\caption{Two dimensional fitness distribution.}
\end{figure}

The dynamics of the bulk was modeled by a system of ordinary differential equations, in which abundances changed in proportion to their selective advantage (\textbf{eqt. 1}). 
\begin{equation} 
\dot{n}_{ij}(t) = (r_{ij}-\bar{r}) n_{ij}(t). 
\end{equation}
Abundances in the stochastic front were negligible in size relative to $N$ due to our parameter assumptions stated earlier.  Since the bulk represents the majority of population, we assumed that the frequencies of the bulk followed an analogous ODE.
\begin{equation} 
\dot{p}_{ij}(t) = (r_{ij}-\bar{r}) p_{ij}(t) 
\end{equation}
These frequencies determine all of the quantitative genetic measures that are of interest to us.\par

Classes in the stochastic front intermittently transition from stochastic to deterministic behavior and become part of the bulk once they have grown sufficiently large. Such classes are said to have established. Prior to establishment, random fluctuations due to drift drive the dynamics of abundances.  The behavior of these classes can be modeled as a branching process fed by incoming mutants generated in adjacent, less fit, classes within the bulk. Each mutant lineage that appears in the stochastic front either goes establishes or goes extinct. The probability of fixation measures the chances of the former outcome, and it depends on the selective advantage of the mutant. If its genotype is $(i,j)$, then its selective advantage is $s_{ij}=(r_{ij}-\bar{r})$. Standard mathematical results for branching processes can be used to solve for the lineage's probability of fixation which is given by equation 3.
\begin{equation}
\pi_{fix} = \frac{s_{ij}}{ 1+s_{ij}}
\end{equation}
The lineage establishes when it has reached a size if roughly $1/s_{ij}$, and its subsequent growth is exponential. \cite{desai2007beneficial} expressed the abundances of these lineages as $n_{ij}(t) = \frac{1}{s_{ij}} \exp[{s_{ij}(t-\tau)}]$. The random variable $\tau$ is the time of establishment and it captures the cumulative effects of the stochastic fluctuations.  In their initial analysis, Desai and Fisher solve for the probability density function of $\tau$ for lineages destined to establish. In our model, we find it more useful to consider the related random variable $\nu = \frac{1}{s_{ij}}\exp[-\Delta r_{ij}\tau]$ representing an initial size that produces the deterministic trajectory observed after establishment. Using the expression given by Desai and Fisher for the PDF of $\tau$, we can arrive at the cumulative distribution of  $\nu$ (equation 5). 
\begin{equation}
P(\nu \leq \nu_0) = 1- e^{-\pi_{fix} \nu_0}
\end{equation}
Each class in the stochastic front may consist of many mutant lineages. However, it is the the arrival time of the first mutant lineage destined to establish that determines which class will join the bulk next. Once the genotype of this mutant lineage is obtained, its initial size $\nu$ can be sampled. The class can then be regarded as belonging to the bulk at the determined arrival time with initial size $\nu$. \par
% * <masel@email.arizona.edu> 2017-07-29T16:52:06.887Z:
% 
% > This was obtained using the expression below derived by \citep{Uecker2011} [eqt. 16b].
% Do you use the full generality of Uecker & Hermisson, or are earlier expressions, eg as used in Desai & Fisher, sufficient for your work?
% 
% 
% ^ <kgomez81@math.arizona.edu> 2017-08-04T23:43:18.217Z.
In order to investigate how of traits affect one another, we identified the contributions of each trait to the evolution of the population. The rate of adaptation can obtained by differentiating the population's mean relative fitness $\bar{r}=\sum_{ij} p_{ij} r_{ij}$, which is itself the sum of the two marginal means, $\bar{r_1}=\sum_{ij} p_{ij} is_1$ and $\bar{r_2}=\sum_{ij} p_{ij} j s_2$.  It follows that the total rate of adaptation is sum of rates of adaptation corresponding to each trait ($\dot{\bar{r}}=\dot{\bar{r_1}} +\dot{\bar{r_2}}$).  We can apply Lande's result to the two components $\dot{\bar{r_1}} $ and $\dot{\bar{r_2}} $ and rewrite them in terms of the trait variances $\sigma_k^2$ (trait $k=1,2$) and their covariance $\sigma_{12}$.  In doing so, we obtain the system expressed in equation 6.
% * <masel@email.arizona.edu> 2017-07-29T17:21:52.605Z:
% 
% >  We found that each of these rates could then be rewritten in terms of trait variances and covariance
% This is not something that "we found" or "obtained". It is simply a statement of Lande's equation for an appropriate (normalized) definition of what a trait is.  Also, remember to define the various sigma terms.
% 
% ^.
% * <masel@email.arizona.edu> 2017-07-29T17:18:22.431Z:
% 
% > arrived at the expression $\dot{\bar{r}}=\dot{\bar{r_1}} +\dot{\bar{r_2}}$
% I'm OK going straight to the expression. Your derivation of it is hard to follow, but I'm not convinced it is necessary.
% 
% ^.
\begin{equation}
\left(
\begin{array}{c}
\dot{\bar{r_1}} \\
\dot{\bar{r_2}} 
\end{array}
\right)
=
\left(
\begin{array}{cc}
\sigma_1^2 & \sigma_{12} \\
\sigma_{12} & \sigma_2^2 
\end{array}
\right)
\left(
\begin{array}{c}
1 \\
1 
\end{array}
\right)
\end{equation}
The components of the selection gradient and trait response have been normalized in the expression above.  

Our numerical implementation consisted of two routines which ran iteratively for the duration of the simulation. Each iteration began with solving for the time-dependent abundances in the bulk governed by \textbf{equation 1} over a time interval of one-thousand generations. We used Mathematica's standard numerical ODE solver to obtain the solutions and used them to determine the appearance time of the next mutant lineage destined to establish. Having computed the timing of this event, we evaluated the solutions and accordingly updated the abundances of the bulk. Any classes with less than one individual were removed.  The establishing class was then added to the bulk with an appropriate initial size. Following this step, the next iteration began with solving for the abundances of the new bulk, followed by the subsequent steps outlined above.\par

The contributions of the stochastic front were incorporated into the adaptive evolution of the population by simulating each establishment event.  We did this by identifying the first mutant lineage destined to establish among all mutants appearing in the stochastic front.  Mutants appear according to a time-inhomogeneous Poisson process with rate functions given by the abundances of the adjacent classes from which they originated. Consequently, the arrival time $\tau_l$ of the $l^{th}$ mutant generated from the class $n_{ij}(t)$ having a mutation in trait $k$ can be used to form the random variable
\[ 
\int_0^{\tau_l} U_k n_{ij}(t) dt ,
\] 
which has the distribution of $l$ i.i.d exponentially distributed summed random variables. By randomly sampling from the this distribution, we could then solve for each arrival time $\tau_l$.   We then computed the probability of fixation using \textbf{equation 5}.  The integral was numerically approximated up to an appropriate cutoff.  To determine if a mutation would establish, we sampled a uniform random variable and checked it against the calculated $\pi_{fix}$. The next class to establish was chosen accordingly and incorporated into the bulk with a random initial size generated with from \textbf{equation 4}.\par

Simulations were initialized with monoclonal populations at carrying capacity consisting of individuals having no beneficial mutations in either trait and relative fitness set zero. We allowed the simulation to run for 5,000 generations to allow the population to achieve beneficial mutation-selection balance before collecting data. This ensured that the transient effects from the initializing would not distort the statistics collected for the distribution. Following the burn period of the simulation, we recorded the existing classes in the bulk and their respective abundances every one 100 generations and intermittent times at which new classes established. The data was used to compute the marginal mean fitness and variances of each trait, as well as their covariance. From these quantities we also computed the instantaneous rate of adaptation (\textbf{eqtn. 5}), the mean rate of fitness increase in each trait and the overall rate of adaptation.

\section{Results and Discussion}

\subsection{Results}
Our simulations were carried out for the symmetric case with parameters $s$ and $U$ being identical for both traits, but in collecting our data, we regarded trait one as the focal trait. The population parameters $N$, $s$, and $U$ were chosen using the results \citep{desai2007beneficial} for fitness variance. Specifically, we ensured that the two traits would have a sufficient amount of genetic diversity to study how their evolution is affected by interactions with the other trait ($\sigma_k^2(N,s_k,U_k) > s_k^2$ ).  We also focused on parameter sets that were in agreement with reported values of $N$, $s$, and $U$ in evolutionary experiments involving model asexual micro-organisms, or .  We measured the fitness variance of trait one over time, as well as its covariance, and then calculated the contribution of each to the reduction in trait one's rate of adaptation. 

In the one dimensional traveling wave model, $N$, $s$, and $U$  prescribe a specific

We also selected parameters that coincided with in evolutionary experiments for micro-organisms that typically are regarded to be asexual, or have very limited amounts of sex.  In table 1 we summarize these ranges for various micro-organisms of interest, these 

\begin{table*}[ht]
\centering
\begin{tabular}{ | c | c | c | c |}
\hline Micro-organism& Population Size ($N$)	& Selection Coefficient ($s$) & Beneficial Mutation Rate ($U$) \\
\hline \begin{tabular}{c} \textit{E}. coli \\ \citep{Perfeito2007} \end{tabular} & $10^6 - 10^{14}$ & $0.5 \% - 2 \%$  & $10^{-9} - 10^{-5}$ \\
\hline \begin{tabular}{c} \textit{S}. cerevisiae \\ \citep{desai2007speed,Levy2015} \end{tabular} & $10^7 - 10^{12}$ & $0.5 \%- 3 \%$  & $10^{-5} - 10^{-4}$ \\
\hline
\end{tabular}
\caption{Population parameters measured in evolution experiments for key microorganisms. Beneficial mutations rates are given in units of mutations per genome per generation.} 
\label{Table 1}
\end{table*}

\begin{figure}
\includegraphics[width=1\linewidth]{fig2a_N-10p09_c1-0d01_c2-0d01_U1-1x10pn5_U2-1x10pn5_exp1.pdf}
\label{Figure 2.}
\caption{Trait 1 variance and covariance.}
\end{figure}
Figure 1 also demonstrates that the covariance over time fluctuates 

\subsection{Discussion}

In our model, covariance between traits is arises entirely from linkage disequilibrium. We can assume that specific alleles at either locus correspond to the number of beneficial mutations that have accumulated in the corresponding trait.  It follows that the $p_{ij}$ are the gametic frequencies for the genome consisting of allele $i$ paired with allele $j$, and $D_{ij} = p_{ij}-p_{i\cdot} p_{\cdot j}$ is their coefficient of linkage disequilibrium ( $p_{i\cdot}$ and $p_{\cdot j}$ are marginal distributions). By applying \textbf{equation 2}, we can derive a mathematical relationship between covariance and linkage disequilibrium.
% * <masel@email.arizona.edu> 2017-07-29T17:34:47.794Z:
% 
% > specific alleles at either locus correspond to the number of beneficial mutations that have accumulated in the corresponding trait
% Multiple loci can contribute to the same trait. An allele cannot be a "number of beneficial mutations", because an allele must pertain to one locus. Stick to the word "genotype" not "allele", eg "genotype ij has i beneficial mutations in trait 1 and j beneficial mutations in trait 2 compared to the ancestral population". This means that you cannot calculate linkage disequilibrium as you do: linkage disequilibrium is generally defined as being between two particular beneficial mutations. What you calculated was simply a covariance. Because of this confusion, I got a bit lost here and so skipped Equation 6.
% 
% ^.
\begin{equation}
\sigma_{12}=\sum_{ij}D_{ij}[(i-\bar{i})s_1][(j-\bar{j})s_2]. 
\end{equation} 
As the population adapts, mutations and selection  increase and decrease the amounts LD which ultimately determine how traits interact in their evolution.\par

We examined the resulting dynamics of trait evolution using numerical simulations of our model using Mathematica. We focused on modeling the symmetric case, in which the two traits had identical mutation rate and selection coefficients. These, along with the population size, were chosen to match values reported in evolutionary experiments involving relevant micro-organisms (see Table 1). 
% \section{In-text Citations}

% Add citations using the \verb|\citep{}| command, for example \citep{neher2013genealogies} or for multiple citations, \citep{neher2013genealogies, rodelsperger2014characterization}

% \section{Examples of Article Components}
% \label{sec:examples}

% The sections below show examples of different header levels, which you can use in the primary sections of the manuscript (Results, Discussion, etc.) to organize your content.

% \section{First level section header}

% Use this level to group two or more closely related headings in a long article.

% \subsection{Second level section header}

% Second level section text.

% \subsubsection{Third level section header:}

% Third level section text. These headings may be numbered, but only when the numbers must be cited in the text. 

% \section{Figures and Tables}

% Figures and Tables should be labelled and referenced in the standard way using the \verb|\label{}| and \verb|\ref{}| commands.

% \subsection{Sample Figure}

% Figure \ref{fig:spectrum} shows an example figure.

% \begin{figure}[htbp]
% \centering
% \includegraphics[width=\linewidth]{example-figure}
% \caption{Example figure from \url{10.1534/genetics.114.173807}. Please include your figures in the manuscript for the review process. You can upload figures to Overleaf via the Project menu. Upon acceptance, we'll ask for your figure files to be uploaded in any of the following formats: TIFF (.tiff), JPEG (.jpg), Microsoft PowerPoint (.ppt), EPS (.eps), or Adobe Illustrator (.ai).  Images should be a minimum of 300 dpi in resolution and 500 dpi minimum if line art images.  RGB, CMYK, and Grayscale are all acceptable. Halftones should be high contrast with sharp detail, because some loss of detail and contrast is inevitable in the production process. Figures should be 10-20 cm in width and 1-25 cm in height. Graph axes must be exactly perpendicular and all lines of equal density.
% Label multiple figure parts with A, B, etc. in bolded type, and use Arrows and numbers to draw attention to areas you want to highlight. Legends should start with a brief title and should be a self-contained description of the content of the figure that provides enough detail to fully understand the data presented. All conventional symbols used to indicate figure data points are available for typesetting; unconventional symbols should not be used. Italicize all mathematical variables (both in the figure legend and figure) , genotypes, and additional symbols that are normally italicized.  
% }%
% \label{fig:spectrum}
% \end{figure}


% \subsection{Sample Table}

% Table \ref{tab:shape-functions} shows an example table. Avoid shading, color type, line drawings, graphics, or other illustrations within tables. Use tables for data only; present drawings, graphics, and illustrations as separate figures. Histograms should not be used to present data that can be captured easily in text or small tables, as they take up much more space.  

% Tables numbers are given in Arabic numerals. Tables should not be numbered 1A, 1B, etc., but if necessary, interior parts of the table can be labeled A, B, etc. for easy reference in the text.  


% \begin{table*}[htbp]
% \centering
% \caption{\bf Students and their grades}
% \begin{tableminipage}{\textwidth}
% \begin{tabularx}{\textwidth}{XXXX}
% \hline
% Student & Grade\footnote{This is an example of a footnote in a table. Lowercase, superscript italic letters (a, b, c, etc.) are used by default. You can also use *, **, and *** to indicate conventional levels of statistical significance, explained below the table.} & Rank & Notes \\
% \hline
% Alice & 82\% & 1 & Performed very well.\\
% Bob & 65\% & 3 & Not up to his usual standard.\\
% Charlie & 73\% & 2 & A good attempt.\\
% \hline
% \end{tabularx}
%   \label{tab:shape-functions}
% \end{tableminipage}
% \end{table*}

% \section{Sample Equation}

% Let $X_1, X_2, \ldots, X_n$ be a sequence of independent and identically distributed random variables with $\text{E}[X_i] = \mu$ and $\text{Var}[X_i] = \sigma^2 < \infty$, and let
% \begin{equation}
% S_n = \frac{X_1 + X_2 + \cdots + X_n}{n}
%       = \frac{1}{n}\sum_{i}^{n} X_i
% \label{eq:refname1}
% \end{equation}
% denote their mean. Then as $n$ approaches infinity, the random variables $\sqrt{n}(S_n - \mu)$ converge in distribution to a normal $\mathcal{N}(0, \sigma^2)$.

\bibliography{biblio}

\end{document}